La qualità finale di un prodotto è determinata in maniera decisiva dalla qualità dei processi che portano alla produzione dello stesso, per questo motivo il team ha deciso di adottare lo standard ISO/IEC 15504 conosciuto anche come \citgl{SPICE} il quale definisce un modello di valutazione dello stadio di maturità di un processo. SPICE prevede sei livelli di maturità per un dato processo e per ognuno di essi definisce anche degli attributi di processo che permettono di determinare con maggior precisione se un livello di maturità viene raggiunto o meno dal processo in questione. In questo documento verranno citate solo le voci ritenute rilevanti in relazione al contesto.
\\\\
\textbf{Level 0 - Incomplete process:} il processo non è implementato o non raggiunge gli obiettivi prefissati, gli output del processo sono pochi o inesistenti. Gli attributi di tale processo sono:
\begin{itemize}
    \item non sono previsti attributi di processo per questo livello.
\end{itemize}
\\\\
\textbf{Level 1 - Performed process:} il processo viene attuato e raggiunge gli obiettivi prefissati, tuttavia non viene scrupolosamente controllato. Gli attributi di tale processo sono: 
\begin{itemize}
    \item \textbf{P.A. 1.1 - Performed process:} capacità di raggiungere i propri obiettivi e produrre output identificabili.
\end{itemize}
\\\\
\textbf{Level 2 - Managed process:} il processo viene attuato, controllato, tracciato e gli output prodotti raggiungono standard prefissati. Gli attributi di tale processo sono:
\begin{itemize}
    \item \textbf{P.A. 2.1 - Performance management:} capacità di produrre output che raggiungono gli obiettivi prefissati;
    \item \textbf{P.A. 2.2 - Work product management:} capacità di produrre output controllato e tracciato.
\end{itemize}
\\\\
\textbf{Level 3 - Established process:} il processo viene attuato e controllato seguendo i principi dell'ingegneria del software. Gli attributi di tale processo sono:
\begin{itemize}
    \item \textbf{P.A. 3.1 - Process definition:} capacità di produrre output che si attengano agli standard dell'ingegneria del software;
    \item \textbf{P.A. 3.2 - Process resource:} capacità di produrre output efficacemente utilizzando una quantità di risorse ragionevole.
\end{itemize}
\\\\
\textbf{Level 4 - Predictable process:} il processo viene attuato con vincoli determinati a raggiungere gli obbiettivi previsti. Il processo risulta essere ben collaudato nella pratica. Gli attributi di tale processo sono:
\begin{itemize}
    \item \textbf{P.A 4.1 - Process measurement:} capacità di utilizzare le misure ottenute durante l'esecuzione del processo per verificare in futuro il raggiungimento degli obiettivi prefissati;
    \item \textbf{P.A. 4.2 Process control:} capacità di modificare l'esecuzione del processo in seguito ai dati raccolti
\end{itemize}
\\\\
\textbf{Level 5 - Optimizing process:} il processo ha una certa consistenza nel raggiungere i propri obiettivi, viene ottimizzato per adempiere al meglio agli obiettivi correnti e futuri. gli attributi di tale processo sono:
\begin{itemize}
    \item \textbf{P.A. 5.1 - Process change:} capacità di tracciare tutti i cambiamenti del processo, siano essi strutturati o di esecuzione;
    \item \textbf{P.A. 5.2 - Continuous improvement:} capacità di implementare le modifiche applicate.
\end{itemize}
\\
Per tutti gli attributi di processo, SPICE fornisce un metodo di valutazione per misurare il loro raggiungimento:
\begin{itemize}
    \item \textbf{N:} non posseduto (0\% - 15\%);
    \item \textbf{P:} parzialmente posseduto (16\% - 50\%);
    \item \textbf{L:} largamente posseduto (51\% - 85\%);
    \item \textbf{F:} totalmente posseduto (86\% - 100\%).
\end{itemize}
\\
Tali valori possono essere utilizzati nel ciclo \citgl{PDCA} il cui scopo è quello di controllare la qualità di un processo durante tutto il suo ciclo di vita e permettere il miglioramento in efficacia ed efficienza dello stesso. Le fasi descritte da PDCA sono le seguenti:
\begin{itemize}
    \item \textbf{Plan:} fase di pianificazione dove si decidono e si individuano gli obiettivi di qualità e i risultati desiderati;
    \item \textbf{Do:} fase in cui si mette in atto il piano stabilito nella fase precedente;
    \item \textbf{Check:} fase di vendita in cui si confrontano i dati in output della fase Do con i risultati previsti in fase Plan;
    \item \textbf{Act:} fase in cui si individuano le cause delle eventuali discordanze riscontrate in fase di Check e si determinano le azioni da intraprendere per risolvere tali discordanze e migliorare il processo aumentandone cosi la qualità.
\end{itemize}
\\\\
Il gruppo ha inoltre individuato nello standard ISO/IEC 12207 alcuni processi il cui scopo è quello di garantire la qualità del prodotto finale. In questo documento verranno citate solo le voci ritenute rilevanti in relazione al contesto.

\subsection{PROC[0001] - Project assessment and Control Process}
Lo scopo di questo processo è quello di determinare lo stato del lavoro svolto ed assicurare che il tutto si stia svolgendo secondo i piani ed entro i limiti di risorse e tempo prestabiliti.
\subsubsection{Obiettivi}
\begin{itemize}
    \item Ogni membro del gruppo svolgerà il \citgl{task} assegnatogli nei tempi previsti;
    \item Le risorse impiegate per una fase non dovranno superare i limiti prestabiliti.
\end{itemize}

\subsubsection{Strategie} 
Il \citgl{Project Manager} deve monitorare lo svolgimento del processo in modo da rilevare il prima possibile eventuali ritardi nello svolgimento dei task e/o un utilizzo delle risorse superiore ai limiti prestabiliti. In caso di rilevamento di tale eccedenza, il gruppo dovrà assolutamente risolvere il problema entro la data prevista per la consegna finale del prodotto.

\subsubsection{Metriche}
\paragraph{M[PROC][0001] - Schedule Variance} 
\begin{itemize}
    \item \textbf{Range di accettazione}: >= 0
    \item \textbf{Range ottimale}: >= 0
\end{itemize}

\paragraph{M[PROC][0002] - Budget Variance} 
\begin{itemize}
    \item \textbf{Range di accettazione}: >= 0
    \item \textbf{Range ottimale}: >= 0
\end{itemize}

\subsection{PROC[0002] - Risk Management Process}
Lo scopo di questo processo è quello di individuare, analizzare e monitorare i rischi durante l'intera durata del progetto.

\subsubsection{Obiettivi}
\begin{itemize}
    \item Il gruppo individuerà i rischi nella prima fase del progetto e ne terrà traccia fino a quando il rischio non sarà più una possibile evenienza;
    \item L'individuazione del rischi verrà svolta ad ogni fase in modo da identificare nuovi possibili rischi introdotti dalle attività svolte nella fase precedente.
\end{itemize}

\subsubsection{Strategie}
Il gruppo dovrà tenere sempre sotto stretta osservazione tutti i rischi in modo da poter mitigare al meglio l'eventuale manifestazione.

\subsubsection{Metriche}
\paragraph{M[PROC][0003] - Rischi non individuati}
\begin{itemize}
    \item \textbf{Range di accettazione}: 0 - 4
    \item \textbf{Range ottimale:} 0
\end{itemize}

\subsection{PROC[0003] - Software Detailed Design Process}
Lo scopo di questo processo è fornire una progettazione di dettaglio del prodotto che andrà ad implementare i requisiti individuati.

\subsubsection{Obiettivi}
\begin{itemize}
    \item Il grado di dettaglio della progettazione deve fornire sufficiente informazione per procedere alla codifica e testing di un'unità senza bisogno di ulteriori informazioni.
\end{itemize}

\subsubsection{Strategie}
Le componenti individuate durante l'analisi verranno suddivise in piccole unità codificabili e testabili facilmente.

\subsubsection{Metriche}

    \paragraph{M[PROC][0004]: Numero campi dati per classe}
\begin{itemize}
    \item \textbf{Range di accettazione}: 0-16
    \item \textbf{Range ottimale}: 0-10
\end{itemize}

\paragraph{M[PROC][0005] - Metodi per classe}
\begin{itemize}
    \item \textbf{Range di accettazione}: 1-10
    \item \textbf{Range ottimale}: 1-7
\end{itemize}

\paragraph{M[PROC][0006] - Parametri per metodo} 
\begin{itemize}
    \item \textbf{Range di accettazione}: 0-8
    \item \textbf{Range ottimale}: 0-5
\end{itemize}


\paragraph{M[PROC][0007]: Grado di instabilità}
\begin{itemize}
    \item \textbf{Range di accettazione}: 0-0.8
    \item \textbf{Range ottimale}: 0.3-0.7
    \end{itemize}
    
\subsection{PROC[0004] - Software Construction Process}
Lo scopo di questo processo è definire le attività principali volte alla produzione di unità software.

\subsubsection{Obiettivi}
\begin{itemize}
    \item il codice prodotto dovrà risultare di bassa complessità per facilitarne la verifica e la complessità;
    \item sdoppiamenti di flusso verranno ridotti al minimo necessario;
    \item il codice prodotto dovrà risultare facilmente manutenibile.
\end{itemize}

\subsubsection{Strategie}
Il team si impegna a mantenere una complessità bassa nella stesura del codice.

\subsubsection{Metriche}
\paragraph{M[PROC][0008] - Complessità Ciclomatica}
\begin{itemize}
    \item \textbf{Range di accettazione}: 1-15
    \item \textbf{Range ottimale}: 1-10
\end{itemize}

\paragraph{M[PROC][0009] - Linee di codice per linee di comando} 
\begin{itemize}
    \item \textbf{Range di accettazione}: >=25
    \item \textbf{Range ottimale}: >=30
\end{itemize}

\paragraph{M[PROC][0010] - Halstead Difficulty per-function}
\begin{itemize}
    \item \textbf{Range di accettazione}: 0-25
    \item \textbf{Range ottimale:} 0-10
\end{itemize}

\paragraph{M[PROC][0011] - Halstead Volume per-function} 
\begin{itemize}
    \item \textbf{Range di accettazione}: 20-1500
    \item \textbf{Range ottimale:} 20-1000
\end{itemize}

\paragraph{M[PROC][0012] - Halstead Effort per-function} 
Rappresenta il costo necessario a scrivere il codice di una funzione.
\begin{itemize}
    \item \textbf{Range di accettazione}: 0-400
    \item \textbf{Range ottimale:} 0-300
\end{itemize}

\paragraph{M[PROC][0013] - Indice di manutenibilità}
\begin{itemize}
    \item \textbf{Range di accettazione}: 100-171
    \item \textbf{Range ottimale:} 120-171
\end{itemize}

\subsection{PROC[0005] - Test}

\subsubsection{Scopo} 
Lo scopo è di definire una misurazione per l'esecuzione dei test e relativi fallimenti
\paragraph{M[PROC][TM][0001] - Percentuale di codice coperto da test}
\begin{itemize}
    \item \textbf{Range di accettazione}: >=0\%
    \item \textbf{Range ottimale}: 100\%
\end{itemize}

\paragraph{M[PROC][TM][0002] - Percentuale di test passati}
\begin{itemize}
    \item \textbf{Range di accettazione}: 100\%
    \item \textbf{Range ottimale}: 100\%
\end{itemize}

\paragraph{M[PROC][TM][0003] - Percentuale di test non passati}
\begin{itemize}
    \item \textbf{Range di accettazione}: 0\%;
    \item \textbf{Range ottimale}: 0\%.
\end{itemize}

