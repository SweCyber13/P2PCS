\documentclass[a4paper, 12pt]{article}
\usepackage{style}
\usepackage{hyperref}


%colori per tablle tabu
\definecolor{tableHeader}{RGB}{211, 47, 47}
\definecolor{tableLineOne}{RGB}{245, 245, 245}
\definecolor{tableLineTwo}{RGB}{224, 224, 224}
%define header tabelle tabu
\newcommand{\tableHeaderStyle}{
    \rowfont[c]{\leavevmode\color{white}\bfseries}
    \rowcolor{tableHeader}
}


\title{VerbInt20.03}
\author{Cyber13}
\date{March 2019}

\begin{document}
	\begin{titlepage}
		\centering Università degli Studi di Padova
		\line(1,0){350}\\
		\vspace{1.2cm}
		\logo
		\vspace{1.0cm}
		\centering{\bfseries\LARGE Verbale interno  del 20/03/2019\\}
		\vspace{0.5cm}
		\centering{\slshape\large Gruppo Cyber13 - Progetto P2PCS\\}
		\vspace{0.5cm}
		\centering{\bfseries Informazioni sul documento \\}
		\line(1,0){240}\\
		% compilare i campi per ogni documento
		\begin{tabular}{r|l}
			{\textbf{Versione}} 		& 1.0.0\\
			{\textbf{Data Redazione}} 	& 20/03/2019\\	% aggiornare la data
			{\textbf{Responsabile}} 	& Matteo Squeri\\	% aggiornare la data
			{\textbf{Redazione}} 		& Elena Pontecchiani\\ 
			{\textbf{Verifica}} 			& Ilaria Rizzo\\ 
			{\textbf{Approvazione}} 		& Matteo Squeri\\
			{\textbf{Uso}} 				& Interno\\
			{\textbf{Destinatari}} 	& Cyber13\\ & Prof. Tullio Vardanega\\ & Prof. Riccardo Cardin\\
			{\textbf{Mail di contatto}} 	& swe.cyber13@gmail.com\\
		\end{tabular}\\
	\end{titlepage}


\newpage
		\subfile{DiarioModifiche.tex}
	
	\newpage
		\tableofcontents
	    	\newpage
        	\section{Informazioni sulla riunione}
\begin{itemize}
	\item \textbf{Luogo della riunione:} Aula Luf1, Via Luzzatti, Padova;
	\item \textbf{Data della riunione:} 6 Marzo 2019;
	
	\item \textbf{Partecipanti della riunione:}
		\begin{itemize}
		    \item Bira Daniel Mirel;
		    \item Casagrande Andrea;
            \item Garavello Fabio;
            \item Pontecchiani Elena;
			\item Rizzo Ilaria;
			\item Squeri Matteo.
		\end{itemize}
\end{itemize}
	
	
\newpage
\section{Ordine del giorno}
La riunione si è concentrata sugli aspetti riguardanti l'analisi dei requisiti relativamente al materiale recentemente fornito da parte del proponente. In particolare, sono stati discussi i seguenti punti:
\begin{enumerate}
    \item Analizzare nel dettaglio e discutere gli spunti per i requisiti ricavati dallo studio del materiale fornito dal proponente;
	\item Stabilire che modifiche apportare ai requisiti già individuati al fine di adeguarli alle richieste del proponente;
	\item Stabilire in quale misura utilizzare il codice dell'applicazione \citgl{Android} \citgl{Gaiago} messo a disposizione dal proponente;
	\item Stabilire che approccio utilizzare per gestire l'aspetto della \citgl{gamification} per un maggior coinvolgimento dell'utente nelle funzionalità offerte dall'applicativo che il gruppo sta sviluppando.
\end{enumerate}

\newpage
\section{Resoconto}
Al termine della riunione si è pervenuti alle seguenti conclusioni:	
\begin{enumerate}
	\item Dopo una analisi delle richieste del proponente si è deciso di scartare l'idea iniziale di realizzare una chat specifica tra utenti per organizzare la funzionalità di prenotazione dei mezzi rilegando la comunicazione degli utenti allo scambio di email. Questa decisione è derivata dall'eccessiva complessità che questa funzionalità avrebbe aggiunto allo sviluppo e dalla mancanza di specifiche precise a riguardo da parte del proponente.
	\item Si è deciso di utilizzare il servizio di Amazon \citgl{AWS} per gestire le operazioni di registrazione e autenticazione degli utenti all'applicativo. Questa scelta è derivata dalla presenza di codice specifico già funzionante nella documentazione fornita dal proponente e dalla maggiore sicurezza e praticità che questo servizio può offrire all'utente;
    \item Si è deciso di non utilizzare l'intera applicazione come base per la progettazione e sviluppo dell'applicativo, tuttavia il gruppo ha osservato che alcune sezioni di codice possono risultare utili per lo sviluppo di alcune funzionalità individuate nei requisiti. In particolare la natura \citgl{test driven} dello sviluppo dell'applicazione fornita è stata ritenuta di particolare interesse per capire come implementare tale pratica.
    \item Sono state apportate delle modifiche ai requisiti relativi alla gestione dei dati personali dell'utente, decidendo di rimuovere quelli non strettamente necessari all'ambito del \citgl{car sharing}. In particolare si è discusso sugli aspetti che l'utente che offre il mezzo in affitto vuole conoscere dell'utente che lo vuole affittare e viceversa.
    \item In seguito ad una ulteriore visione della documentazione utilizzata in ambito gamification e del materiale fornito dal \citgl{proponente} si è deciso di puntare maggiormente su tre degli specifici \citgl{Core Drives} del framework \citgl{Octalysis}. Questi Core Drives sono:
    \begin{itemize}
        \item Development \& Accomplishment
        \item Social Influence \& Relatedness
        \item Epic Meaning \& Calling
    \end{itemize}
	\item Il gruppo ha deciso che fosse necessario contattare il proponente per chiarire alcuni aspetti riguardanti le decisioni fatte e per constatare che queste si attengano ai requisiti richiesti.
\end{enumerate}
\newpage
\end{document}

    