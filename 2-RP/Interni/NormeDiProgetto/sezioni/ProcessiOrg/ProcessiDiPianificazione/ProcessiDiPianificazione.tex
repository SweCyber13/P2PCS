\subsubsection{Scopo}
In questa sezione vengono elencate le norme relative ai processi di pianificazione interni al gruppo.
\subsubsection{Aspettative}
Il gruppo si aspetta di organizzare in maniera ottimale ed evitando incomprensioni di varia natura la divisione del lavoro da svolgere. Per fare vengono stabiliti ruoli precisi all'interno del progetto.


\subsubsection{Ruoli di progetto}
Nel corso del progetto ogni componente del team avrà modo di ricoprire a rotazione il ruolo di ognuna delle figure aziendali tipicamente coinvolte nella realizzazione di un software.
\\Tale assegnazione cercherà di distribuire nel modo più omogeneo possibile il tempo che ciascun membro investirà in ogni ruolo, dando in ogni caso priorità alla continuità delle attività già in corso.
\\Il Responsabile di Progetto di turno dovrà inoltre garantire che non si creino situazioni di conflitto di interesse che potrebbero compromettere la qualità del prodotto finale.
    \paragraph{Responsabile di progetto}
    ~\\Il "Responsabile di progetto", o "Project Manager", accentra le responsabilità di scelta e approvazione del gruppo, del quale è anche il rappresentante presso proponenti e committenti.
    \\Si occupa quindi di gestire le risorse umane e di coordinare i membri del team, dell'approvazione dei documenti, della pianificazione e delle relazioni esterne.
    \paragraph{Ammininstratore}
    ~\\L'amministratore è una figura di supporto che si occupa di fornire al team gli strumenti per lavorare in maniera rigorosa e regolamentata.
    \\Tali strumenti consistono nella redazione e attuazione di piani e procedure di Gestione della Qualità, la redazione delle Norme di Progetto, il versionamento e la configurazione dei prodotti.
    \\Pur non essendo sotto sua diretta responsabilità, collabora alla redazione del Piano di Progetto e si assicura che la documentazione sia corretta e verificata.
    \paragraph{Analista}
    ~\\L'analista si occupa di analizzare e capire appieno il dominio del problema per identificarne i requisiti espliciti ed impliciti, e per riconoscere ed evitare gravi problemi di progettazione successivi.
    \\Essendo colui che all'interno del team meglio conosce il problema e i requisiti che la sua realizzazione richiede, esso è coinvolto anche nella definizione degli accordi contrattuali, nella verifica delle implicazioni di costo e qualità, nella realizzazione dello Studio di Fattibilità e dell'Analisi dei Requisiti.
    \paragraph{Progettista}
    ~\\Il progettista si occupa della definizione di una soluzione soddisfacente per tutti gli \citgl{stakeholder}, perseguendo efficienza ed efficacia nella soddisfazione dei requisiti a partire dal lavoro dell'Analista.
    \\In tal senso il progettista conosce e applica soluzioni di lavoro note e ottimizzate (\citgl{Best practice}), e si occupa di organizzare e suddividere il sistema in parti di complessità trattabile, per rendere il lavoro di codifica realizzabile e verificabile.
    \paragraph{Programmatore}
    ~\\Il programmatore si occupa della codifica del progetto, realizzando quindi il prodotto finale attraverso l'implementazione dell'architettura definita dal Progettista.
    \\Il codice prodotto deve essere documentato, versionato e mantenibile secondo le norme fissate.
    \\Si occupa anche della scrittura del Manuale Utente e di realizzare le componenti necessarie alla verifica e alla validazione del codice.
    \paragraph{Verificatore}
    ~\\Il verificatore è coinvolto in ogni fase del progetto. Il suo compito è appunto quello di verificare la conformità dei prodotti ai requisiti fissati di funzionalità e qualità.
    \\Il suo coinvolgimento è richiesto in ogni fase poiché deve accertarsi che ad ogni attività di processo eseguita non siano stati introdotti errori.
    \\Attraverso la redazione di un Piano di Qualifica si occupa di illustrare le verifiche effettuate sul progetto e i relativi esiti, secondo quanto previsto dal Piano di Progetto.
    \paragraph{Rotazione dei ruoli}
    ~\\Per assicurare un buon esito del progetto e allo stesso tempo assicurare una rotazione dei ruoli assegnati a ciascun membro del gruppo durante la sua realizzazione, sono necessarie alcune semplice regole:
    \begin{itemize}
        \item Considerare impegni e interessi di ogni singolo membro del gruppo;
        \item Evitare conflitti di interesse, in particolare evitare che un ruolo assegnato ad un membro del gruppo comporti la verifica di lavoro svolto in precedenza dallo stesso mentre ricopriva un altro ruolo all'interno della stessa fase dello sviluppo.
        \item Garantire che al cambio di ruolo il nuovo assegnatario venga posto nelle migliori condizioni di lavoro possibile. In tal senso il precedente assegnatario deve assicurarsi di non lasciare al nuovo criticità da lui non gestibili. Inoltre si occuperà di lasciargli al nuovo una lista di consigli e procedure utili in un documento interno informale.
        \item Garantire che ciascun membro ricoprirà di volta in volta il ruolo a lui assegnato in modo esclusivo: ciascuno dovrà occuparsi di quanto è di sua competenza, senza interferire o incrociare incarichi e responsabilità.
    \end{itemize}
    \subsubsection{Ticketing}
    
    \paragraph{Task list}
    ~\\Lo svolgimento del progetto prevede la suddivisione del modello di sviluppo in varie fasi. Le attività da svolgere in una fase sono contenute in una \citgl{task board}.Il responsabile ha il compito di creare le task board per ogni fase del modello su \citgl{GitHub}, utilizzando il nome "X", dove X identifica in modo significativo la sezione di progetto cui si riferisce. Per esempio, per quanto riguarda i documenti, X segue la nomenclatura indicata nella sezione 3.1.4.1. 
    \paragraph{Task}
     ~\\Ogni singola \citgl{task} viene creata o approvata dal Responsabile di progetto ed è caratterizzata da un titolo significativo per la componente di progetto cui si riferisce. 
    \paragraph{Ticket}
    ~\\Il \citgl{ticket} sono lo strumento attraverso cui viene gestito il ciclo di vita del task, dalla sua definizione, alla sua assegnazione ad uno o più membri del gruppo, alla sua implementazione, verifica, approvazione e chiusura.
     L'assegnazione del \citgl{ticket} ad uno specifico membro del gruppo può avvenire secondo due modalità:
    \begin{itemize}
        \item \textbf{Direttamente}: L'assegnazione del task è già noto al momento della creazione dello stesso. In questo caso il \citgl{ticket} viene assegnato direttamente allo specifico componente del gruppo da parte del Responsabile di progetto.
        \item \textbf{Indirettamente}: Nel caso in cui un membro non è presente agli incontri oppure il lavoro che richiede una \citgl{task} è impegnativo, può non essere possibile assegnare direttamente un ticket ad un componente del gruppo. In questo caso il membro non presente è obbligato a controllare la sezione \citgl{issues} presente su \citgl{GitHub} dove troverà il lavoro da svolgere. Questi tasks inoltre possono essere assegnati qualora il componete finisse i \citgl{ticket} a suo carico.
    \end{itemize}

\newpage
    