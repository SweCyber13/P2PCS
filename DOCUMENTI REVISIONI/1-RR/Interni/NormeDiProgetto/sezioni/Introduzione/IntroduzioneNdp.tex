\subsection{Scopo del documento}
Il documento ha lo scopo di fissare precise norme che regolamenteranno
l’intero svolgimento del progetto. \\
Tutti i membri del team sono obbligati a visionare tale documento e a sottostare alle norme ivi descritte.
Le norme contenute nel documento si occupano di regolamentare i diversi processi all'interno del progetto. La regolamentazione tramite norme permette di sviluppare prodotti coerenti e consistenti tra i vari componenti del gruppo e facilitare l’esecuzione delle  operazioni di verifica.
In particolare le norme si occuperanno di regolamentare:
    \begin{itemize}
        \item Le modalità di interazione tra i membri del team e le persone esterne al team;
        \item L'organizzazione del team: verranno definiti e assegnati ruoli ai membri del team. Per ogni ruolo verranno identificate determinate mansioni;
        \item Le convenzioni tipografiche e le modalità di stesura delle documentazioni;
        \item Gli ambienti di sviluppo, il \citgl{repository} e il \citgl{ticketing};
    \end{itemize}

Nel caso in cui ve ne sia necessità, ogni membro del team potrà contattare il
\citgl {Project Manager} per suggerire eventuali cambiamenti
\subsection{Scopo del prodotto}
Lo scopo del prodotto è quella di realizzare un'applicazione \citgl{Android} che implementi un servizio di car sharing \citgl{peer-to-peer}.

\subsection{Glossario}
Onde evitare ambiguità o incomprensioni di natura lessicale, si allega il \G.
All'interno del documento saranno presenti parole di ambito specifico, uso raro che potrebbero creare incomprensioni. Per una maggiore leggibilità tali parole sono riconoscibili all'interno dei vari documenti in quanto scritte in corsivo e con un 'g' a pedice tra barre orizzontali (per esempio \citgl{Glossario})
Nel caso esistano più ripetizioni di una stessa parola del glossario all'interno di uno stesso paragrafo, la dicitura con la lettera 'g' a pedice sarà inserita solo la prima volta che la parola comparirà.
Il comando LaTeX da utilizzare per contrassegnare un termine da glossario all’interno dei documenti è \textbackslash \textit{citgl}\{..\}.



\subsection{Riferimenti}
    \subsubsection {Riferimenti normativi}
        \begin{itemize}
        \item \NdP;
    \end{itemize}
    \subsubsection {Riferimenti informativi}
        \begin{itemize}
        \item \AdR;
        \item \PdP;
        \item \PdQ;
        \item \SdF;
        \item Sito del corso di Ingegneria del Software: \\ \url{https://www.math.unipd.it/~tullio/IS-1/2018/}
        \item  Software Engineering - Ian Sommerville - 10th Edition;
        \item Materiale suggerito dalla proponente:
        \begin{itemize}
            \item Materiale \citgl{MVP}:
           \\ \href{https://medium.com/@cervonefrancesco/model-view-presenter-android-guidelines-94970b430ddf}{https://model-view-presenter-android-guidelines}
            \item Materiale Testing e \citgl{Refactoring}
           \\ \url{https://www.youtube.com/watch?v=_NnElPO5BU0}
        \end{itemize}
        
    \end{itemize}
