\subsubsection{Scopo}
In questa sezione vengono elencate direttive, linee guida e strumenti che il gruppo dovrà utilizzare a fini organizzativi all'interno dello svolgimento del lavoro.
\subsubsection{Aspettative}
Il gruppo si attende alle seguenti direttive al fine di organizzare, evitando incomprensioni di varia natura, gli incontri tra i componenti.



\subsubsection{Comunicazione}
In questa sezione vengono illustrati i metodi di comunicazione sia tra i membri all'interno del gruppo Cyber13 sia tra il gruppo e le entità esterne, come Committenti e Proponenti.
    \paragraph{Comunicazioni interne}
    ~\\Le comunicazioni interne al gruppo avvengono tramite \citgl{Slack}, strumento di messaggistica nel quale sono stati predisposti dei canali tematici, suddivisi per argomento:
    \begin{itemize}
    \item \textbf{general:} Canale di carattere generale, per coordinare luoghi e orari di ritrovo, scambiare opinioni sul progetto e in generale inviare comunicati che non trovano spazio in alcuno degli altri canali specifici.
    \item \textbf{documentirr:} Canale predisposto per la coordinazione e l'aggiornamento dei lavori del team su tutti i documenti da approntare in vista della RR di Aprile 2019 (si procederà allo stesso modo anche per le fasi successive).
    
    \end{itemize}
    \paragraph{Comunicazioni esterne}
    ~\\Alcune norme regolano anche le comunicazioni con le parti esterne al gruppo Cyber13, nello specifico:
    \begin{itemize}
    \item La proponente GaiaGo rappresentata da Filippo De Pretto, con la collaborazione del quale si definiranno i \citgl{requisiti} che porteranno alla realizzazione del prodotto;
    \item I committenti Prof. Tullio Vardanega e Prof. Riccardo Cardin, ai quali verrà fornita la documentazione richiesta in ciascuna revisione di progetto. 
    \end{itemize}
    \textbf{Comunicazioni esterne scritte}
    ~\\ Le comunicazioni esterne scritte vengono effettuate utilizzando l'indirizzo e-mail del gruppo \url{swe.cyber13@gmail.com} a cui tutti i membri hanno accesso.
    Per comunicare con il Proponente viene utilizzato l'indirizzo e-mail \url{f.depretto@gaiago.com}.
    Anche le comunicazioni con i committenti avvengono esclusivamente tramite l'indirizzo e-mail del gruppo: ogni messaggio deve essere definito da un oggetto chiaro e conciso, così come deve essere il suo contenuto, e rivolgersi ai committenti in modo formale e cortese.
    
\subsubsection{Riunioni}
Le riunioni interne ed esterne al team Cyber13 sono un momento fondamentale per il corretto avanzamento e conclusione del progetto. Alcune regole ne normano quindi l'organizzazione e lo svolgimento.
\\Per ogni riunione viene nominato un Segretario tra i membri del gruppo, il quale garantirà il rispetto dell'ordine del giorno, si occuperà di annotare i punti più importanti discussi e da tali appunti redigerà infine un Verbale di Riunione. Quest'ultimo non è un ruolo fisso, quindi il membro del team che ne ricoprirà la carica potrà variare da riunione a riunione se necessario.
\\Le riunioni che non potranno avere un luogo fisico di ritrovo avverranno in remoto attraverso il programma \citgl{Skype} o lo strumento \citgl{ Google Meet}, secondo quanto concordato preventivamente prima dell'appuntamento.

    \paragraph{Riunioni interne}
    ~\\La partecipazione è ammessa ai soli membri del team Cyber13.
    \\Ordine del giorno, data, ora e luogo della riunione vengono decisi dal Responsabile di progetto, che si occupa anche di approvare il Verbale di Riunione stilato dal Segretario.
    \\I membri del team hanno il dovere di presentarsi in orario all'appuntamento, comunicando eventuali ritardi e partecipando attivamente e costruttivamente alla discussione.
    \\La riunione è considerata valida se almeno 5 dei 6 membri del team sono presenti. In circostanze eccezionali qualora un membro fosse impossibilitato ad essere presente di persona  ma la sua presenza dovesse essere fondamentale per la discussione dell'ordine del giorno, esso potrà prendere parte alla riunione da remoto attraverso uno strumento di videochiamata come Skype o Google Meet.
    
    \paragraph{Riunioni esterne}
    ~\\Prendono parte a queste riunioni i membri del team Cyber13 e la Proponente.
    \\Essendo la sede di GaiaGo a Milano, le riunioni esterne avranno il più delle volte luogo da remoto, attraverso uno strumento quale Skype o Google Meet, deciso di volta in volta prima dell'appuntamento.
    \\Le riunioni esterne cui la Proponente prenderà parte di persona saranno invece effettuate presso la Torre Tullio Levi Civita (ex Torre Archimede), previa disponibilità dei locali.
    \\Come per le riunioni interne, il Verbale di Riunione sarà redatto da un Segretario eletto di volta in volta e sarà approvato in seguito dal Responsabile di Progetto.
    \\Data l'elevata importanza delle riunioni esterne, esse vengono considerate valide anche alla presenza di un singolo membro del team Cyber13 e della Proponente.
    
    \paragraph{Verbale di riunione}
    ~\\Il Segretario eletto di volta in volta si occupa di stilarlo secondo il seguente schema:
    \begin{itemize}
        \item \textbf{Verbale TIPO del DATA:} Il frontespizio del Verbale di Riunione permette di identificarne il TIPO (Interno/Esterno) e la DATA in cui la riunione ha avuto luogo.
        \item \textbf{Diario delle modifiche:} Come definito nella sezione 3.1.6.1.
        \item \textbf{Indice:} Semplice indice dei contenuti del Verbale.
        \item \textbf{Informazioni sulla riunione:} Deve contenere obbligatoriamente le seguenti informazioni:
        \\ \textbf{- Luogo e data della riunione:} ad esempio Padova, 15 Marzo 2019;
        \\ \textbf{- Partecipanti alla riunione:} a partire dai Proponenti/Committenti in caso di Riunione Esterna, seguiti dai nomi dei membri del team Cyber13 presenti elencati in ordine alfabetico.
        \\\\Possono invece considerarsi informazioni facoltative ma utili:
        \\ \textbf{- Ora di inizio e di fine:} scritte nel formato 24h;
        \\ \textbf{- Motivo della riunione:} Illustra in forma riassuntiva i motivi generali della riunione;
        \item \textbf{Ordine del giorno:} Indica attraverso un elenco numerato gli argomenti da discutere così come definiti dal Responsabile di Progetto;
        \item \textbf{Resoconto:} riassunto di quanto discusso in riunione, redatto dal Segretario. Indicherà chiaramente quali punti dell'Ordine del Giorno sono stati effettivamente affrontati, eventuali argomenti discussi che non fossero presenti nell'Ordine del Giorno e le relative decisioni prese.
        \\Il tracciamento delle decisioni prese avverrà attraverso un sistema a codici del tipo VER-DATA.X dove VER significa Verbale, DATA è sostituito con la data della riunione e X è un numero sequenziale che identificherà univocamente la decisione cui si sta facendo riferimento.
    \end{itemize}
    \\\\ \textbf{Nomenclatura e archiviazione dei verbali:} I Verbali di Riunione saranno archiviati con la nomenclatura VER-DATA dove DATA è la data della riunione nel formato YYYY-MM-DD per permetterne un posizionamento semplice e cronologico all'interno della cartella assegnata.
    \\Trattandosi di documenti esterni e ufficiali, verranno archiviati nella cartella di Repository dedicata e saranno redatti in LaTeX.