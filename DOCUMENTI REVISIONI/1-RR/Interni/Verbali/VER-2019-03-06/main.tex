\documentclass[a4paper, 12pt]{article}
\usepackage{style}
\usepackage{hyperref}


%colori per tablle tabu
\definecolor{tableHeader}{RGB}{211, 47, 47}
\definecolor{tableLineOne}{RGB}{245, 245, 245}
\definecolor{tableLineTwo}{RGB}{224, 224, 224}
%define header tabelle tabu
\newcommand{\tableHeaderStyle}{
    \rowfont[c]{\leavevmode\color{white}\bfseries}
    \rowcolor{tableHeader}
}



\title{VerbInt06.03}
\author{Cyber13}
\date{March 2019}

\begin{document}
	\begin{titlepage}
		\centering Università degli Studi di Padova
		\line(1,0){350}\\
		\vspace{1.2cm}
		\logo
		\vspace{1.0cm}
		\centering{\bfseries\LARGE Verbale interno  del 06/03/2019\\}
		\vspace{0.5cm}
		\centering{\slshape\large Gruppo Cyber13 - Progetto P2PCS\\}
		\vspace{0.5cm}
		\centering{\bfseries Informazioni sul documento \\}
		\line(1,0){240}\\
		% compilare i campi per ogni documento
		\begin{tabular}{r|l}
			{\textbf{Versione}} 			& 1.0.0\\
			{\textbf{Data Redazione}} 	& 06/03/2019\\	% aggiornare la data
			{\textbf{Responsabile}} 	& Elena Pontecchiani\\	% aggiornare la data
			{\textbf{Redazione}} 		& Matteo Squeri\\ 
			{\textbf{Verifica}} 			& Fabio Garavello \\ 
			{\textbf{Approvazione}} 		& Elena Pontecchiani\\
			{\textbf{Uso}} 				& Interno\\
			{\textbf{Destinatari}} 	& Cyber 13\\ & Prof. Tullio Vardanega\\ & Prof. Riccardo Cardin\\
			{\textbf{Mail di contatto}} 	& swe.cyber13@gmail.com\\
		\end{tabular}\\
	\end{titlepage}

	\newpage
\newpage
		\subfile{DiarioModifiche.tex}
	
	\newpage
		\tableofcontents
	    	\newpage
        	\section{Informazioni sulla riunione}
\begin{itemize}
	\item \textbf{Luogo della riunione:} Aula Luf1, Via Luzzatti, Padova;
	\item \textbf{Data della riunione:} 6 Marzo 2019;
	
	\item \textbf{Partecipanti della riunione:}
		\begin{itemize}
		    \item Bira Daniel Mirel;
            \item Garavello Fabio;
            \item Pontecchiani Elena;
			\item Rizzo Ilaria;
			\item Squeri Matteo.
		\end{itemize}
\end{itemize}


	
	
	
\newpage
\section{Ordine del giorno}
Durante la riunione sono stati discussi i seguenti punti:
\begin{enumerate}
	\item Stabilire il nome e il logo del gruppo;
	\item Stabilire strumenti di lavoro da adottare;
	\item Stabilire capitolato da svolgere.
\end{enumerate}
	

\newpage
\section{Resoconto}
Al termine della riunione si è pervenuti alle seguenti conclusioni:	
\begin{enumerate}
	\item Sono stati decisi il nome del gruppo di lavoro (Cyber13) e il relativo logo;
	\item Gli strumenti di lavoro che si sono decisi di adottare sono i seguenti:
		\begin{itemize}
			\item Account Google: swe.cyber13@gmail.com;
			\item Piattaforma di versionamento: GitHub;
			\item Discussione in merito agli strumenti di stesura di documenti e scelta Overleaf per il collaborazione online per la collaborazione di documenti;
			\item Linguaggio di markup per la preparazione di testi per i documenti: Latex;
			\item Strumento di comunicazione tra i membri del gruppo: Slack.
		\end{itemize}
	\item Per effettuare la scelta del capitolato da svolgere ogni componente ha espresso una lista dei tre capitolati preferiti in ordine crescente di preferenza. In seguito si sono valutate le varie liste e si è concluso che la maggioranza del gruppo era indirizzata verso il progetto P2PCS proposto dall'azienda \citgl{GaiaGo}. 
\end{enumerate}
\newpage
\end{document}

    