\subsection{Consuntivo e preventivo a finire}
In questa sezioni verranno riportati i consuntivi delle varie fasi con delle conclusioni sullo stato dello sviluppo. Viene inoltre allegato un preventivo a finire che terrà conto delle sole fasi le cui ore vengono rendicontate. \\
Nei consuntivi vengono riportati:
\begin{itemize}
    \item le ore e i costi preventivati;
    \item le ore e i costi effettivi;
\end{itemize}l
Viene inserito tra parentesi, a fianco dei costi e delle ore effettive, un valore che rappresenta la differenza rispetto a quanto preventivato:  
\begin{itemize}
    \item \textbf{Positivo}: Se le ore effettive sono minori rispetto a quelle preventivate o se il costo è diminuito.
    \item \textbf{Negativo}: Se le ore effettive sono maggiori rispetto a quelle preventivate o se il costo è aumentato.
\end{itemize}
Se non sono presenti valori tra parentesi significa che il numero di ore e i costi per i ruoli sono state come preventivate.
\subsubsection{Consuntivi fasi di Analisi e Analisi di dettaglio}
Essendo entrambe le fasi precedenti alla Revisione dei Requisiti, vengono considerate assieme. La consegna del materiale è fissata in data 12-04-2019, durante la fase di Analisi di dettaglio, è seguita poi dalla sola preparazione della presentazione, la quale non comporta l'impiego di specifici ruoli con conseguenti costi economici, perciò il consuntivo non tiene conto del periodo che va dal 13-04-2019 al 19-04-2019.

\newpage

\paragraph{Consuntivo fase di Analisi}
La seguente tabella contiene i dati sia orari che economici del consuntivo per la fase di Analisi:

\begin{table}[H]
\taburowcolors[2] 2{tableLineOne .. tableLineTwo}
\tabulinesep = 10pt
\everyrow{\tabucline[.4mm  white]{}}
\begin{tabu} to \textwidth { X[c,1.2] X[c] X[c] X[c,1.1] X[c]}
    \tableHeaderStyle
    Ruolo & Ore preventivate & Ore effettive & Costi preventivati in \euro & Costi effettivi in \euro \\
    Responsabile & 25 & 22 (+3) & 720,00 & 630,00 (+90)\\
    Amministratore & 29 & 25 (+4) & 580,00 & 500,00 (+80)\\
    Progettista &  &  &  & \\
    Programmatore &  &  &  & \\
    Verificatore & 38 & 35 (+3) & 570,00 & 525 (+45)\\
    Analista & 44 & 50 (-6) & 1.100,00 & 1.250,00 (-150) \\
    \textbf{Totale} & \textbf{136} & \textbf{132 (+4)} & \textbf{3.000,00} & \textbf{2.905,00 (+95)}  \\
\end{tabu}
\caption{Consuntivo fase di Analisi}
\end{table}
Resoconto:
\begin{itemize}
    \item \textbf{Differenza oraria}: +4 ore;
    \item \textbf{Differenza costi}: +95 \euro ;
\end{itemize}
Non sono stati commessi dei ritardi durante l’esecuzione delle varie attività previste.

\newpage

\paragraph{Consuntivo fase di Analisi di dettaglio}
La seguente tabella contiene i dati sia orari che economici del consuntivo per la fase di Analisi di dettaglio:

\begin{table}[H]
\taburowcolors[2] 2{tableLineOne .. tableLineTwo}
\tabulinesep = 10pt
\everyrow{\tabucline[.4mm  white]{}}
\begin{tabu} to \textwidth { X[c,1.2] X[c] X[c] X[c,1.1] X[c]}
    \tableHeaderStyle
    Ruolo & Ore preventivate & Ore effettive & Costi preventivati in \euro & Costi effettivi in \euro \\
    Responsabile & 7 & 7 & 210,00 & 210,00\\
    Amministratore & 9 & 6 (+3) & 180,00 &  120,00 (+60)\\
    Progettista &  &  &  & \\
    Programmatore &  &  &  & \\
    Verificatore & 12 & 20 (-8) & 180,00 & 300,00 (-120,00)\\
    Analista & 17 & 11 (+6) & 425,00 & 275,00 (+150)  \\
    \textbf{Totale} & \textbf{45} & \textbf{44 (+1)} & \textbf{995,00} & \textbf{905,00 (+90)}  \\
\end{tabu}
\caption{Consuntivo fase di Analisi di dettaglio}
\end{table}
Resoconto:
\begin{itemize}
    \item \textbf{Differenza oraria}: +1 ore;
    \item \textbf{Differenza costi}: +90 \euro ;
\end{itemize}
Non sono stati commessi dei ritardi durante l’esecuzione delle varie attività previste.

\newpage 

\subsubsection{Considerazioni}

Durante la fase di Analisi si è rivelato necessario un maggiore utilizzo della figura di Analista rispetto a quanto preventivato, dovuto principalmente alla vasta quantità di informazioni messe a disposizione nelle prime settimane da parte della \citgl{proponente} \citgl{GaiaGo} per quanto riguarda l'individuazione dei requisiti dell'applicazione. Tuttavia si è riusciti a risparmiare qualche ora nei ruoli di Responsabile ed Amministratore, la cui necessità è stata sovrastimata. Data la maggior necessità della figura di Analista in questa fase, per quel che riguarda riguarda la figura di Verificatore si è deciso di spostare una parte dei processi di verifica nella fase successiva, soprattutto per l'attività Analisi dei requisiti. Le ore di verifica si sono tuttavia dimostrate adeguate per le altre attività svolte. In conclusione il risultato del periodo è stato positivo, con quattro ore lavorative e 95 \euro{} in meno rispetto a quanto preventivato. \\\\
Per quel che riguarda la fase di Analisi di dettaglio, le ore del ruolo di Responsabile si sono rivelate adeguate per l'approvazione dei documenti ai fini del rilascio in Revisione dei Requisiti. Per quanto concerne il ruolo di Amministratore è stato nuovamente sovrastimato il conteggio orario nel preventivo. Come conseguenza degli scelte effettuate nella fase di Analisi, in quella di Analisi di dettaglio si è ridimensionata la presenza della figura di Analista e ampliata quella di Verificatore, a cui è stato richiesto un maggior impiego nella verifica del documento Analisi dei Requisiti. In conclusione il risultato del periodo è stato discretamente positivo in quanto il quantitativo di ore di lavoro complessivo è rimasto sostanzialmente invariato, con la differenza di una sola ora e 90 \euro{} in meno rispetto a quanto preventivato.

\subsubsection{Preventivo a finire}
Attualmente il preventivo a finire non presenta variazioni dato che le fasi considerate nel consuntivo, Analisi ed Analisi di dettaglio, sono considerate di solo investimento. Questa sezione sarà aggiornata successivamente al termine di ciascuna delle fasi rimanenti.