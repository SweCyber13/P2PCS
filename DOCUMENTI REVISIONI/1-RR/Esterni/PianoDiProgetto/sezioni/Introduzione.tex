\subsection{Scopo del documento}
Lo scopo del documento è quello di pianificare accuratamente la suddivisione per lo svolgimento delle attività dei membri del gruppo Cyber13. In particolare il documento conterrà:
\begin{itemize}
    \item Una breve descrizione del modello progettuale utilizzato per lo sviluppo, correlata dalla motivazione che ne ha portato alla scelta;
    \item L'analisi dei rischi che potrebbero presentarsi durante lo sviluppo;
    \item L'organizzazione e la suddivisione dei tempi per lo svolgimento di ogni attività;
    \item Una stima preventiva per l'utilizzo delle risorse;
    \item Un calcolo del consuntivo di utilizzo delle risorse durante lo svolgimento del progetto.
\end{itemize}
\subsection{Scopo del prodotto}
Lo scopo del prodotto è quella di realizzare un'applicazione \citgl{Andorid} che implementi un servizio di \citgl{car sharing} \citgl{peer-to-peer} con dinamiche di \citgl{gamification}.
\subsection{Glossario}
Onde evitare ambiguità o incomprensioni di natura lessicale, si allega il \G.
All'interno del documento saranno presenti parole di ambito specifico, uso raro che potrebbero creare incomprensioni. Per una maggiore leggibilità tali parole sono riconoscibili all'interno dei vari documenti in quanto scritte in corsivo e con un 'g' a pedice tra barre orizzontali (per esempio \citgl{Glossario})
\subsection{Scadenze}
Il documento si baserà sulle scadenze di seguito riportate, che il gruppo Cyber13 ha deciso di rispettare:
\begin{itemize}
    \item \textbf{Revisione dei Requisiti}: 19-04-2019;
    \item \textbf{Revisione di Progettazione}: 17-05-2019;
    \item \textbf{Revisione di Qualifica}: 17-06-2019;
    \item \textbf{Revisione di Accettazione}: 15-07-2019.
\end{itemize}
\subsection{Riferimenti}
    \subsubsection {Riferimenti normativi}
        \begin{itemize}
        \item \NdP;
        \item Regolamento organigramma: \\
        \url{https://www.math.unipd.it/~tullio/IS-1/2018/Progetto/RO.html}.
        \end{itemize}
    \subsubsection {Riferimenti informativi}
        \begin{itemize}
        \item \AdR;
        \item \PdQ;
        \item Sito del corso di Ingegneria del Software \url{https://www.math.unipd.it/~tullio/IS-1/2018/};
        \item  Software Engineering 10th edition) Ian Sommerville,Pearson Education, Addison-Wesley;
    \end{itemize}