\documentclass[a4paper, oneside, openany, dvipsnames, table]{article}
\usepackage[utf8]{inputenc}


\usepackage{lmodern}
\usepackage{breakurl}
\usepackage[T1]{fontenc}
\usepackage[italian]{babel}
\usepackage{color, colortbl}
\usepackage{style}


%colori per tablle tabu
\definecolor{tableHeader}{RGB}{211, 47, 47}
\definecolor{tableLineOne}{RGB}{245, 245, 245}
\definecolor{tableLineTwo}{RGB}{224, 224, 224}
%define header tabelle tabu
\newcommand{\tableHeaderStyle}{
    \rowfont[c]{\leavevmode\color{white}\bfseries}
    \rowcolor{tableHeader}
}

\title{Verb-2019-03-12}
\author{Cyber13}
\date{March 2019}

\begin{document}
\begin{titlepage}
		\centering Università degli Studi di Padova
		\line(1,0){350}\\
		\vspace{1.2cm}
		\logo
		\vspace{1.0cm}
		\centering{\bfseries\LARGE Verbale esterno del 12/03/2019 \\}
		\vspace{0.5cm}
		\centering{\slshape\large Gruppo Cyber13 - Progetto P2PCS\\}
		\vspace{0.5cm}
		\centering{\bfseries Informazioni sul documento \\}
		\line(1,0){240}\\
		% compilare i campi per ogni documento
		\begin{tabular}{r|l}
			{\textbf{Versione}} 			& 1.0.0\\
			{\textbf{Data Redazione}} 	& 12/03/2019\\	% aggiornare la data
			{\textbf{Responsabile}} 	& Elena Pontecchiani\\	% aggiornare la data
			{\textbf{Redazione}} 		& Ilaria Rizzo\\ 
			{\textbf{Verifica}} 			& Fabio Garavello\\ 
			{\textbf{Approvazione}} 		& Elena Pontecchiani\\
			{\textbf{Uso}} 				& Interno\\
			{\textbf{Destinatari}} & GaiaGo\\ 	& Cyber13\\ & Prof. Tullio Vardanega\\ & Prof. Riccardo Cardin\\
			{\textbf{Mail di contatto}} 	& swe.cyber13@gmail.com\\
		\end{tabular}\\
	\end{titlepage}
	
	
\newpage
		\subfile{DiarioModifiche.tex}



\newpage
\tableofcontents
\newpage
\section{Informazioni sulla riunione}
\begin{itemize}
	\item \textbf{Luogo della riunione:} Aula 1BC45 Torre Tullio Levi Civita, Via Trieste, Padova;
	\item \textbf{Data della riunione:} 12 Marzo 2019;
	
	\item \textbf{Partecipanti della riunione:}
		\begin{itemize}
			\item Bira Daniel Mirel;
			\item Casagrande Andrea;
			\item Garavello Fabio;
			\item Pontecchiani Elena;
			\item Rizzo Ilaria;
			\item Squeri Matteo.
		\end{itemize}
\end{itemize}


	
	
	
\newpage
\section{Ordine del giorno}
Si è svolta una chiamata con l'azienda proponente del progetto per chiarire i seguenti punti:
\begin{enumerate}
	\item Elenco dettagliato delle tecnologie di utilizzo richieste;
	\item Chiarimenti in merito l' \citgl{IDE di sviluppo} necessario per lo sviluppo Android;
	\item Chiarimenti sulla piattaforma \citgl{Movens};
	\item Chiarimenti in merito all'utenza che utilizzerà l'applicazione. Nello specifico si richiede alla proponente se è necessario differenziare tra un utente privato e un'azienda;
	\item Delucidazioni in merito ai dati che l'applicazione dovrà richiedere ad un utente al fine di registrarlo nella piattaforma.
	 
		
\end{enumerate}
	

\newpage
\section{Resoconto}
Al termine della riunione si è pervenuti alle seguenti conclusioni:	
\begin{enumerate}
	\item L'azienda non ha posto nessun obbligo sull'utilizzo delle tecnologie ma ha consigliato le seguenti:
		\begin{itemize}
			\item Android studio: IDE per lo sviluppo Android;
			\item Kotlin: linguaggio di sviluppo per l'applicazione;
			\item AWS: piattaforma cloud;
			\item Bitrise: ambiente di test;
			\item Espresso: framework per l'esecuzione di test UI;
			\item API Google Maps: per lo sviluppo di funzionalità di geo localizzazione.
			
		\end{itemize}
	\item  Son state discusse varie tematiche riguardanti la \citgl{gamification}. L'azienda proponente ha lasciato ampia libertà in merito all'implementazione di meccaniche di gioco, dando suggerimenti utili a semplificare il lavoro;
	\item È stato mostrato a video il funzionamento di della piattaforma Movens;
	\item È stato imposto di usare il modello Octalysis per l'implementazione della gamification;
	\item È stata suggerito l'utilizzo \citgl{Figma} come piattaforma per i \citgl{mockup}.
	\item Si è discussa l'importanza dei \citgl{test UI} per verificare le funzionalità dell'interfaccia utente.  
\end{enumerate}
\newpage
\end{document}