\section{A}
\subsection*{Ambiente di sviluppo} Programma o insieme di programmi che viene utilizzato dai programmatori per lo sviluppo del codice sorgente di un programma.
\subsection*{Android} Sistema operativo open source per dispositivi mobili sviluppato da Google Inc. e basato sul kernel Linux.
\subsection*{Android studio} IDE di sviluppo specifico per applicazioni Android fornito direttamente da Google Inc. Disponibile gratuitamente sotto licenza Apache 2.0.
\subsection*{API} Insieme di procedure (in genere raggruppate per strumenti specifici) atte all'esecuzione di un dato compito. Possono essere librerie software proprie di un linguaggio di programmazione oppure di un software che fornisca servizi implementabili in programmi differenti.
\subsection*{API Google Maps} Strumenti che permettono di ottenere informazioni sulla posizione geografica e permettono la visualizzazione di mappe sia su software desktop che su app mobile.
\subsection*{Architettura} Struttura complessiva del sistema in termini dei principali moduli di cui esso è composto e delle relazioni macroscopiche fra di essi.
\subsection*{AWS} Insieme di servizi di cloud computing che compongono la piattaforma on demand offerta dall'azienda Amazon. Tra i servizi offerti vi sono framework di sviluppo, database cloud, API di riconoscimento vocale e molto altro.
\newpage
\section{B}
\subsection*{Backend} Parte del sistema nascosta all'utente finale del prodotto software ma fondamentale poichè permette l'effettivo funzionamento. Solitamente è gestita da alcuni membri preposti dell'azienda che ha realizzato l'applicativo.
\subsection*{Best Practice} Con questo termine si intende  indicare la formalizzazione di pratiche applicate in un ambito specifico che esperienze precedenti hanno dimostrato essere le migliori per efficacia ed efficienza.
\subsection*{Bitrise} Ambiente di test automatici che permette di sviluppare seguendo le pratiche della Continuous Integration e della Test Driven Development.
\subsection*{Blockchain} Una sottofamiglia di tecnologie in cui il registro è strutturato come una catena di blocchi contenenti le transazioni e la cui validazione è affidata a un meccanismo di consenso, distribuito su tutti i nodi della rete, ossia su tutti i nodi che sono autorizzati a partecipare al processo di validazione delle transazioni da includere nel registro.
\subsection*{Branch} Ogni qualvolta nello sviluppo di un software si intraprende l'implementazione di una nuova funzionalità o si procede alla correzione di un bug si apre un nuovo ramo di lavoro (in inglese "Branch") per tracciare in parallelo l'avanzamento di ciascun lavoro. Ogni ramo viene poi unito al branch principale, chiamato Master.
\newpage
\section{C}
\subsection*{Capitolato} É un documento di carattere tecnico usato dalla proponente per definire le specifiche del progetto proposto.
\subsection*{Caso d'uso}
Il caso d'uso in informatica è una tecnica usata nei processi di Ingegneria del Software per effettuare in maniera esaustiva e non ambigua la raccolta dei requisiti al fine di produrre software di qualità.

Essa consiste nel valutare ogni requisito focalizzandosi sugli attori che interagiscono col sistema, valutandone le varie interazioni. 

\subsection*{Car sharing} Servizio di mobilità urbana che permette agli utenti di utilizzare un veicolo su prenotazione noleggiandolo per un periodo limitato di tempo.
\subsection*{Caso d'uso} Tecnica usata nei processi di ingegneria del software per effettuare in maniera esaustiva e non ambigua, la raccolta dei requisiti al fine di produrre software di qualità. Consiste nel valutare ogni requisito focalizzandosi sugli attori che interagiscono col sistema, descrivendone le varie interazioni.
\subsection*{Cloud} Paradigma di erogazione di servizi offerti on demand da un fornitore ad un cliente finale attraverso la rete Internet.
\subsection*{Commit} Modifica a un file contenuto nel repository, richiede un commento descrittivo del tipo di modifica prima di essere pubblicata.
\subsection*{Committente} La figura che commissiona un lavoro, indipendentemente dall'entità o dall'importo.
\subsection*{Continuous Delivery} Pratica adottata nello sviluppo software che segue la Continuous Integration e fa in modo che le funzioni di test e implementazione delle applicazioni siano eseguite in maniera certa e standardizzata, allo scopo di permettere il rilascio di una versione aggiornata del software in modo rapido e sicuro.
\subsection*{Continuous Integration} Pratica adottata nello sviluppo software che consiste nell'allineamento frequente dagli ambienti di lavoro degli sviluppatori verso l'ambiente condiviso.
\subsection*{Core Drive} Termine usato nel framework di gamification Octalysis per indicare gli 8 "nuclei" fondamentali che lo definiscono. Ciascuno di essi descrive un aspetto del comportamento umano su cui fanno leva giochi e videogiochi definendo un insieme di Component specifici.
\newpage
\section{D}
\subsection*{D-apps} Applicazioni decentralizzate che usano contratti intelligenti per il loro trattamento.
\subsection*{Deploy} Implementazione/sviluppo di una soluzione progettata precedentemente.
\subsection*{Design pattern} Una soluzione progettuale generale e standardizzata ad un problema ricorrente nello sviluppo.
\subsection*{Diagrammi dei casi d'uso} Diagrammi UML dedicati alla descrizione delle funzioni o servizi offerti da un sistema, così come sono percepiti e utilizzati dagli attori che interagiscono col sistema stesso.
\subsection*{Dominio applicativo} Contesto in cui una applicazione software opera.
\subsection*{Dominio tecnologico} Insieme di tecnologie utilizzate in un determinato contesto.
\subsection*{Draw.io} Software gratuito online per la realizzazione di diagrammi flowchart, di processi, UML, ER e di rete.
\newpage
\section{E}
\subsection*{E-commerce} Insieme delle transazioni per la commercializzazione di beni e servizi tra produttore/fornitore e consumatore, realizzate tramite la rete Internet.
\subsection*{Efficacia} Capacità di produrre l'effetto ed i risultati voluti o sperati.
\subsection*{Efficienza} Misura dell’abilità di raggiungere l’obiettivo impiegando le risorse minime indispensabili.
\subsection*{ERC20} Standard tecnico utilizzato per i contratti intelligenti sulla blockchain di Ethereum per l'implementazione di token.
\subsection*{Espresso} Framework per l’esecuzione dei test UI all’interno dell’ambiente Bitrise.
\subsection*{Event-driven} Paradigma di programmazione in cui il flusso di esecuzione di un programma è largamente determinato dal verificarsi di eventi esterni, ad esempio un click effettuato col mouse.
\newpage
\section{F}
\subsection*{Farfalla Project} Strumento per il calcolo dell'indice Gulpease di un testo.
\subsection*{Feed RSS} Uno dei più popolari formati per la distribuzione di contenuti Web. Utilizza il formato XML.
\subsection*{Framework} In termini più generici Framework significa "Struttura", ed è un termine utilizzato per indicare tutto ciò che in qualche modo organizza e schematizza uno specifico contesto. In informatica più precisamente indica una architettura logica di supporto su cui un software può essere progettato e realizzato, spesso facilitandone lo sviluppo da parte del programmatore.
\subsection*{Freeware} Si riferisce ai software proprietari ma con utilizzo gratuito.
\subsection*{Frontend} Parte del programma visibile all'utente con cui egli può interagire, definita dalle funzionalità messe a disposizione e dall'interfaccia utente che gli permette di accedervi e farvi uso.
\newpage
\section{G}
\subsection*{GaiaGo} Azienda di sviluppo software proponente del progetto P2PCS.
\subsection*{Gamification} Utilizzo di elementi mutuati dai giochi e delle tecniche di game design in contesti che per loro natura non sono legati a concetti di gioco.
\subsection*{Gantt} Henry Lawrence Gantt, ingegnere statunitense che nel 1917 ideò un diagramma che porta il suo nome utilizzato per la gestione dei progetti.
\subsection*{Git} Software distribuito di controllo di versione multipiattaforma.
\subsection*{GitBash} Client Git a riga di comando per Windows.
\subsection*{Github} Servizio di web hosting per lo sviluppo di progetti software che usa il sistema di controllo di versione Git.
\subsection*{GitKraken} Client per Git che permette di eseguire tramite interfaccia grafica comandi Git.
\subsection*{Google Drive} Servizio web, in ambiente cloud computing, di memorizzazione e sincronizzazione dati online.
\subsection*{Google Meet} Servizio di videoconferenza offerto da Google.
\subsection*{Gulpease} L'Indice Gulpease è un indice di leggibilità di un testo tarato sulla lingua italiana[1]. Rispetto ad altri ha il vantaggio di utilizzare la lunghezza delle parole in lettere anziché in sillabe, semplificandone il calcolo automatico. 
\newpage
\section{H}
\newpage
\section{I}
\subsection*{IDE di sviluppo} Software che, in fase di programmazione, aiuta i programmatori nello sviluppo del codice sorgente di un programma.
\subsection*{Interfaccia web} Interazione con l'ambiente web.
\subsection*{iOS} Sistema operativo proprietario sviluppato da Apple per iPhone, iPod touch e iPad.
\subsection*{ISO} Norme che possono essere applicate in modo uniforme e coerente a livello internazionale.
\subsection*{Issues} Sezione di GitHub dove sono presenti le attività da svolgere.
\newpage
\section{J}
\newpage
\section{K}
\subsection*{Kotlin} Linguaggio di programmazione general purpose, multi-paradigma, open source sviluppato dall'azienda di software JetBrains.
\newpage
\section{L}
\subsection*{LaTeX}  È un linguaggio di markup per la preparazione di documenti. Essendo puramente testuale  permette il versionamento.
\subsection*{Linguaggio Markup} Insieme di regole che descrivono i meccanismi di rappresentazione (strutturali, semantici, presentazionali) di un testo.
\newpage
\section{M}
\subsection*{Machine learning} Abilità delle macchine di apprendere, senza essere state esplicitamente e preventivamente programmate, a partire da un'analisi dei dati messi a disposizione, da cui vengono estratti pattern ricorrenti.
\subsection*{Microsoft Excel} Programma prodotto da Microsoft, dedicato alla produzione ed alla gestione di fogli elettronici.
\subsection*{Mockup} Serve a rendere l’idea del progetto finale ma senza l’interattività di un prototipo, rappresentando nel dettaglio i vari contenuti e le funzionalità base dell’applicazione in maniera statica.
\subsection*{Movens} Piattaforma di sviluppo per funzionalità mobile dedicate alla mobility.
\subsection*{Multipiattaforma} Aggettivo che si riferisce ad un software che si adatta a più piattaforme.
\subsection*{MYSQL} Sistema di gestione di database relazionali composto da un client a riga di comando e un server.
\subsection*{MVP} 
Pattern architetturale in grado di separare la logica di presentazione dei dati dalla logica di business. Il pattern è basato sulla separazione dei compiti fra i componenti software che interpretano tre ruoli principali:
    \begin{itemize}
        \item Il model fornisce i metodi per accedere ai dati utili all'applicazione;
        \item Il view visualizza i dati contenuti nel model e si occupa dell'interazione con utenti e agenti;
        \item Il controller riceve i comandi dell'utente (in genere attraverso il view) e li attua modificando lo stato degli altri due componenti.
    \end{itemize}
    
    
    
\newpage
\section{N}
\newpage
\section{O}
\subsection*{Octalysis} Framework per facilitare la comprensione e l'applicazione della gamification. Composto da una struttura a forma ottagonale che ad ogni lato associa  gli otto elementi chiave, chiamati Core Drives, che motivano e spingono l'essere umano a compiere determinate azioni.
\subsection*{On demand} L'accesso alle risorse informatiche solo quando necessario.
\subsection*{Open source} Software di cui i detentori dei diritti rendono pubblico il codice sorgente.
\subsection*{Overleaf} È uno strumento nel cloud utilizzato per scrivere e pubblicare testi scritti in linguaggio LaTeX.
\newpage
\section{P}
\subsection*{P2PCS} Peer to peer car sharing. Si tratta del nome riferito al progetto scelto dal gruppo.
\subsection*{PDCA} Plan Do Check Act, consiste in un metodo di gestione iterativo composto da quattro fasi (pianificare, fare, verificare, agire) utilizzato per il miglioramento continuo dei processi e dei prodotti
\subsection*{Peer-to-peer} Interazione diretta tra soggetti con eguale livello, senza che vi sia un’unità di controllo/gestione di livello superiore.
\subsection*{PHP} Linguaggio di scripting interpretato, originariamente concepito per la programmazione di pagine web dinamiche.
\subsection*{Plug-in} Modulo aggiuntivo non autonomo di un programma, utilizzato per estenderne o ampliarne le funzioni.
\subsection*{Product Baseline} Punto concordato degli attributi di un prodotto, in un momento specifico, che funge da base per la definizione del cambiamento.
\subsection*{Project Manager} Unico responsabile dell'avvio, pianificazione, esecuzione, controllo e chiusura di un progetto facendo ricorso a tecniche e metodi di project management.
\subsection*{Proponente} La figura che propone un lavoro. Nel nostro caso, l'azienda Gaia Go.
\subsection*{Provisioning} Processo di preparazione e dotazione di una rete per consentire di fornire nuovi servizi ai propri utenti.
\subsection*{Publisher-subscriber} Design pattern utilizzato per la comunicazione asincrona fra diversi processi, oggetti o altri agenti.
\newpage
\section{Q}
\newpage
\section{R}
\subsection*{Refactoring} Tecnica strutturata per modificare la struttura interna di porzioni di codice senza modificarne il comportamento esterno", applicata per migliorare alcune caratteristiche non funzionali del software.
\subsection*{Repository} Un repository (letteralmente deposito o ripostiglio) è un ambiente di un sistema informativo in cui vengono gestiti i metadati.
\subsection*{Requisiti} Ciascuna delle qualità necessarie e richieste per uno determinato scopo.
\subsection*{Reti Bayesiane} Modello grafico probabilistico che rappresenta un insieme di variabili stocastiche con le loro dipendenze condizionali attraverso l'uso di un grafo aciclico diretto.
\subsection*{RR} Revisione dei requisiti.
\newpage
\section{S}
\subsection*{Secondo lotto} Lotto di studenti del corso di Ingegneria del Software che, non avendo soddisfatto i requisiti per accedere alla prima occasione al progetto didattico, hanno potuto iniziarlo solamente in un periodo successivo.
\subsection*{Skill Alexa} Un set di azioni o compiti programmati che possono essere eseguiti dall'assistente vocale Amazon Alexa.
\subsection*{Skype} Software proprietario \citgl{freeware} della Microsoft che fornisce un servizio di telefonate, messaggistica, video chiamate, trasferimento di file. 
\subsection*{Slack} Software di messaggistica multipiattafomra specifico per gruppi di lavoro.
\subsection*{SPICE} Software Process Improvement and Capability Determination. Denominazione attribuita alla valutazione del processo nello standard ISO/IEC 15504, si tratta di un insieme di documenti tecnici che stabiliscono degli standard per il processo di sviluppo del software.
\subsection*{Stakeholder} Soggetto influente nei confronti di un progetto.
\subsection*{Standard} Modello di riferimento a cui ci si uniforma.
\subsection*{Subtask} Incarico da portare a termine per il completamento di un'attività principale di dimensioni maggiori e composta da più subtask.
\newpage
\section{T}
\subsection*{Task} Rappresenta un'attività e/o compito da svolgere.
\subsection*{Task board} Tabella che rappresenta un insieme di attività da svolgere.
\subsection*{Technology Baseline} Punto concordato degli attributi di una tecnologia, in un momento specifico, che funge da base per la definizione del cambiamento.
\subsection*{Template} Modello predefinito che consente di creare o inserire contenuti di diverso tipo in un documento.
\subsection*{Test driven development} Modello di sviluppo del software che prevede che la stesura dei test automatici avvenga prima di quella del software che deve essere sottoposto a test. In tal senso, il software andrà sviluppato in modo da dare esiti positivi ai test e il fatto che ci riesca è garanzia di rispetto dei requisiti fissati.
\subsection*{Test UI} Test automatizzati che eseguono operazioni di test sull'interfaccia utente dell'applicazione.
\subsection*{Ticket} Notifica di segnalazione di una attività da svolgere per l'avanzamento del progetto oppure una segnalazione di un errore all'interno del software da risolvere.
\subsection*{Ticketing} Metodo tramite il quale si tiene traccia e si assegnano determinati compiti ai rispettivi membri del gruppo.
\subsection*{Token} Forma di astrazione di valore impiegato per contare una quantità di utilizzo.
\subsection*{Top down} Strategia di elaborazione dell'informazione e di gestione delle conoscenze ponendo l'attenzione prima sui punti fondamentali per poi scendere nei particolari.
\newpage
\section{U}
\subsection*{UML} Linguaggio di modellazione e specifica basato sul paradigma orientato agli oggetti.
\newpage
\section{V}
\subsection*{Validazione} Attività dalla quale devono risultare gli esiti della verifica. In particolare un prodotto può considerarsi validato una volta che le varie fasi di verifica danno tutte esito positivo.
\subsection*{Verifica} Attività di controllo delle singole fasi di un progetto.
\newpage
\section{W}
\subsection*{Workflow} Flusso di lavoro su un determinato ambito, definito attraverso un set di regole ed ordine di esecuzione ben definito.
\newpage
\section{X}
\newpage
\section{Y}
\newpage
\section{Z}