\documentclass[a4paper, oneside, openany, dvipsnames, table]{article}
\usepackage[utf8]{inputenc}


\usepackage{lmodern}
\usepackage{breakurl}
\usepackage[T1]{fontenc}
\usepackage[italian]{babel}



\begin{document}
\tableofcontents
\newpage
\section{Informazioni sulla riunione}
\begin{itemize}
	\item \textbf{Luogo della riunione:} Aula Luf1, Via Luzzati, Padova;
	\item \textbf{Data della riunione:} 6 Marzo 2019;
	
	\item \textbf{Partecipanti della riunione:}
		\begin{itemize}
			\item Daniel Mirel Bira;
			\item Ilaria Rizzo;
			\item Fabio Garavello;
			\item Elena Pontecchiani;
			\item Matteo Squeri.
		\end{itemize}
\end{itemize}


	
	
	
\newpage
\section{Ordine del giorno}
Durante la riunione sono stati discussi i seguenti punti:
\begin{enumerate}
	\item Stabilire il nome e il logo del gruppo;
	\item Stabilire strumenti di lavoro da adottare;
	\item Stabilire capitolato da svolgere.
		
\end{enumerate}
	

\newpage
\section{Resoconto}
Al termine della riunione si è pervenuti alle seguenti conclusioni:	
\begin{enumerate}
	\item Sono stati decisi il nome del gruppo di lavoro (Cyber13) e il relativo logo;
	\item Gli strumenti di lavoro che si sono decisi di adottare sono i seguenti:
		\begin{itemize}
			\item Account Google: swe.cyber13@gmail.com;
			\item Piattaforma di versionamento: GitHub;
			\item Linguaggio di markup per la preparazione di testi per i documenti: Latex;
			\item Strumento di comunicazione tra i membri del gruppo: Slack.
		\end{itemize}
	\item Per effettuare la scelta del capitolato da svolgere ogni componente ha espresso una lista dei tre capitolati preferiti in ordine crescente di preferenza. In seguito si sono valutate le varie liste e si è concluso che la maggioranza del gruppo era indirizzata verso il progetto P2PCS proposto dall'azienda GaiaGo. 
\end{enumerate}
\newpage
\end{document}