\subsection{Obiettivi del prodotto}
Il prodotto ha come obiettivo l'implementazione di un'applicazione mobile \citgl{Android} in grado di gestire un sistema di condivisione dell'auto "\citgl{peer-to-peer}" e che includa al suo interno meccaniche di gioco. 


\subsection{Funzionalità del prodotto}
Le funzionalità principali offerte dal prodotto sono le seguenti:
\begin{itemize}
    \item Permettere agli utenti, iscritti alla piattaforma e in possesso di un'auto, di mettere in condivisione la propria vettura ad altri utenti;
    \item Permettere agli utenti sprovvisti di vettura di prenotare un mezzo all'interno della piattaforma con il quale compiere un determinato itinerario;
    \item Tutte le ulteriori funzionalità hanno lo scopo di utilizzare meccaniche di \citgl{gamification} affinché l'utente sia motivato a utilizzare l'applicazione.
    
\end{itemize}
\subsection{Caratteristiche degli utenti}
Il prodotto non si rivolge a classi d'utenza appartenenti ad uno specifico ambito: l'utenza dell'applicazione sarà molto generica per età, interessi, ambito sociale e culturale. In particolare l'applicazione riconosce due classi di utenza, non mutuamente esclusive:
\begin{itemize}
    \item Coloro in possesso di un'auto inutilizzata  che condividono attraverso il sistema al fine di ottenere retribuzioni (monetarie o come bonus all'interno dell'applicazione);
    \item Coloro sprovvisti di una vettura e che necessitano di un mezzo per compiere un determinato itinerario.
\end{itemize}

\subsection{Piattaforme di esecuzione}
Il prodotto finale verrà sviluppato su diverse piattaforme d’esecuzione:
\begin{itemize}
    \item  Il \citgl{backend} che può essere sviluppato interamente dal gruppo o utilizzare la piattaforma \citgl{Movens} fornita dal Proponente;
    \item Il \citgl{frontend} deve essere sviluppato su un'applicazione Android già esistente ma non implementata fornita dal Proponente.
\end{itemize}

\subsection{Vincoli}
L'utente, per usufruire del servizio, dovrà possedere un telefono con sistema operativo Android e scaricare l'applicazione dal nome \citgl{GaiaGo}. Dopo aver fatto Registrazione/Login l'utente potrà usufruire dei servizi che l'applicazione fornisce. 
