\subsection{Scopo del documento}
Il presente documento ha lo scopo di presentare un'analisi completa di tutti i \citgl{requisiti} e i \citgl{casi d'uso} individuati dal team Cyber13 per lo sviluppo del progetto \citgl{P2PCS}. La seguente analisi è scaturita sulla base dello studio del capitolato d'appalto e di colloqui con l'azienda proponente \citgl{GaiaGo}.

\subsection{Scopo del prodotto}
Lo scopo del prodotto è di estendere un'applicazione per ambiente Android, già esistente per il \citgl{car sharing} condominiale, a livello \citgl{peer-to-peer}. Inoltre viene richiesto di integrare meccanismi di \citgl{Gamification}, al fine di rendere l'esperienza dell'utente all'interno dell'applicazione più gradevole.

\subsection{Glossario}
Onde evitare ambiguità o incomprensioni di natura lessicale, si allega il \G.
All'interno del documento saranno presenti parole di ambito specifico, uso raro che potrebbero creare incomprensioni. Per una maggiore leggibilità tali parole sono riconoscibili all'interno dei vari documenti in quanto scritte in corsivo e con un 'g' a pedice tra barre orizzontali (per esempio \citgl{Glossario})
\subsection{Riferimenti}
\subsubsection{Riferimenti normativi}
\begin{itemize}
    \item Norme di Progetto: \NdP
    
\end{itemize}
\subsubsection{Riferimenti informativi}
\begin{itemize}
    \item Presentazione del capitolato C5
    \\ \url{https://www.math.unipd.it/~tullio/IS-1/2018/Progetto/C5.pdf}
    \item Presentazione del concetto gamification
    \\ \url{https://www.math.unipd.it/~tullio/IS-1/2018/Progetto/P4.pdf}
    \item Lucidi didattici contenenti materiale didattico affrontato durante il corso di Ingegneria del Software:
        \begin{itemize}
            \item Materiale riguardante i diagrammi UML dei casi d'uso:
            \\ \url{https://www.math.unipd.it/~tullio/IS-1/2018/Dispense/E05b.pdf}
            \item Materiale riguardante il documento "Analisi dei Requisiti":
            \\ \url{https://www.math.unipd.it/~tullio/IS-1/2018/Dispense/L08.pdf}
        \end{itemize}
        
    
\end{itemize}

