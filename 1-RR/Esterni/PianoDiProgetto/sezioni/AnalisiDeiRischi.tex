Di seguito viene effettuata un'approfondita analisi sui rischi che potrebbero compromettere lo svolgimento del progetto. I rischi vengono divisi per fattore di rischio ed analizzati allo scopo di produrre una breve descrizione e misurare la probabilità di occorrenza e la loro gravità. Per ogni rischio individuato viene proposto un piano di contingenza per scongiurarli oppure cercare di ridurre l'impatto sullo sviluppo se inevitabili.
\subsection{Rischi riguardanti i membri del team}

\begin{table}[H]
\taburowcolors[2] 2{tableLineOne .. tableLineTwo}
\tabulinesep = 10pt
\everyrow{\tabucline[.4mm  white]{}}
\begin{tabu} to \textwidth { X[l,1.5] X[l,4] }
    \tableHeaderStyle
    Nome rischio & Scarsa esperienza \\
    Descrizione & Nessun membro del gruppo ha mai lavorato ad un progetto di tale portata. Molte tecnologie e metodologie utilizzate risultano sconosciute a vari membri. \\
    Individuazione & Ciascun componente comunicherà, dove rilevante, le proprie mancanze relativamente ai compiti assegnati. \\
    Occorrenza & Alta \\
    Gravità & Alta \\
    Piano di contingenza & I compiti saranno assegnati ai membri che hanno esperienza in merito. Se ciò non fosse possibile, i membri del gruppo offriranno supporto al componente in difficoltà nel caso specifico. \\
\end{tabu}
\caption{Rischio: Scarsa esperienza}
\end{table}

\begin{table}[H]
\taburowcolors[2] 2{tableLineOne .. tableLineTwo}
\tabulinesep = 10pt
\everyrow{\tabucline[.4mm  white]{}}
\begin{tabu} to \textwidth { X[l,1.5] X[l,4] }
    \tableHeaderStyle
    Nome rischio & Contrasti tra i membri del gruppo \\
    Descrizione & Essendo il gruppo stato formato in modo causale molti membri non hanno mai collaborato ad un progetto. Questa mancanza di conoscenza tra i membri potrebbe portare a tensioni e conflitti sulle modalità di avanzamento dello sviluppo. \\
    Individuazione & Sarà compito del responsabile controllare la collaborazione tra i componenti del gruppo. In caso di problemi con altri componenti, questi saranno comunicati al responsabile se non viene trovata una soluzione indipendentemente. \\
    Occorrenza & Bassa \\
    Gravità & Media \\
    Piano di contingenza & Si cercherà di mantenere separati membri che presentano frequenti contrasti, assegnandoli ad attività diverse.\\
\end{tabu}
\caption{Rischio: Contrasti tra i membri del gruppo}
\end{table}

\begin{table}[H]
\taburowcolors[2] 2{tableLineOne .. tableLineTwo}
\tabulinesep = 10pt
\everyrow{\tabucline[.4mm  white]{}}
\begin{tabu} to \textwidth { X[l,1.5] X[l,4] }
    \tableHeaderStyle
    Nome rischio & Mancanza di disponibilità dei membri \\
    Descrizione & Alcuni membri del gruppo sono studenti lavoratori, quindi il tempo che possono dedicare al progetto è limitato. Inoltre, vista la durata del progetto, questo potrebbe sovrapporsi ad altre attività universitarie a carico dei componenti. \\
    Individuazione & Ogni membro comunicherà con sufficiente anticipo al responsabile i giorni in cui non potrà portare avanti i compiti assegnati.  \\
    Occorrenza & Alta \\
    Gravità & Media \\
    Piano di contingenza & Il carico di lavoro verrà redistribuito tra i membri con maggiore disponibilità di tempo. L'obiettivo rimarrà quello di non far slittare la data di completamento dell'attività. \\
\end{tabu}
\caption {Rischio: Mancanza di disponibilità dei membri}
\end{table}

\begin{table}[H]
\taburowcolors[2] 2{tableLineOne .. tableLineTwo}
\tabulinesep = 10pt
\everyrow{\tabucline[.4mm  white]{}}
\begin{tabu} to \textwidth { X[l,1.5] X[l,4] }
    \tableHeaderStyle
    Nome rischio & Perdita di motivazione \\
    Descrizione &  Data l'estesa durata del progetto è possibile che alcuni membri del gruppo possano perdere motivazione nello svolgimento dei loro compiti, portando a ritardi nello sviluppo.\\
    Individuazione & Sarà compito del responsabile supervisionare il lavoro svolto dai membri del gruppo, notando eventuali ritardi non giustificati nell'avanzamento. \\
    Occorrenza & Bassa\\
    Gravità & Bassa \\
    Piano di contingenza & Il responsabile si occuperà di parlare personalmente con i membri del gruppo in cui ha notato mancanza di impegno nello sviluppo, ricordando loro gli impegni presi.\\
\end{tabu}
\caption{Rischio: Perdita di motivazione}
\end{table}

\subsection{Rischi riguardanti l'organizzazione}

\begin{table}[H]
\taburowcolors[2] 2{tableLineOne .. tableLineTwo}
\tabulinesep = 10pt
\everyrow{\tabucline[.4mm  white]{}}
\begin{tabu} to \textwidth { X[l,1.5] X[l,4] }
    \tableHeaderStyle
    Nome rischio & Sottostima dei costi e dei tempi \\
    Descrizione & Data l'inesperienza in campo organizzativo lungo una durata temporale così ampia da parte del responsabile, è possibile fare delle assunzioni sbagliate nella pianificazione. Può accadere che i tempi dedicati a ciascuna attività si rivelino sottostimati creando quindi problemi nella gestione dell'intero progetto. \\
    Individuazione & Il responsabile verificherà periodicamente lo stato dell'attività e i membri si aggiorneranno costantemente sullo stato dell'avanzamento.\\
    Occorrenza & Media \\
    Gravità & Alta \\
    Piano di contingenza & In caso di ritardi importanti il responsabile modificherà la pianificazione, ridistribuendo il carico lavorativo tra i membri in anticipo sulla data di scadenza per le attività a loro assegnate. \\
\end{tabu}
\caption{Rischio: Valutazione dei costi e dei tempi}
\end{table}

\begin{table}[H]
\taburowcolors[2] 2{tableLineOne .. tableLineTwo}
\tabulinesep = 10pt
\everyrow{\tabucline[.4mm  white]{}}
\begin{tabu} to \textwidth { X[l,1.5] X[l,4] }
    \tableHeaderStyle
    Nome rischio & Comunicazione con la proponente \\
    Descrizione & Durante lo sviluppo il gruppo si impegna a contattare la proponente al fine di stabilire con più chiarezza le funzionalità richieste dal prodotto. Tuttavia potrebbero verificarsi momenti di stallo e incertezza dovuti alla mancanza di disponibilità della proponente, per suoi impegni, di comunicare col team.  \\
    Individuazione & Il responsabile cercherà di stabilire le date per mettersi in contatto con la proponente a seconda della disponibilità di quest'ultima. Al responsabile spetta anche il comunicare al gruppo quando la proponente non risulta più disponibile per una data prefissata. \\
    Occorrenza & Bassa \\
    Gravità & Media \\
    Piano di contingenza & In caso di spostamento di un incontro con la proponente riguardo un argomento di fondamentale importanza per l'avanzamento di alcune attività il responsabile negozierà una nuova data e redistribuirà i membri su attività per le quali non siano attese ulteriori informazioni. \\
\end{tabu}
\caption{Rischio: Comunicazione con la proponente}
\end{table}

\subsection{Rischi riguardanti i requisiti}

\begin{table}[H]
\taburowcolors[2] 2{tableLineOne .. tableLineTwo}
\tabulinesep = 10pt
\everyrow{\tabucline[.4mm  white]{}}
\begin{tabu} to \textwidth { X[l,1.5] X[l,4] }
    \tableHeaderStyle
    Nome rischio & Comprensione dei requisiti \\
    Descrizione & È possibile che durante l'analisi del capitolato alcuni requisiti individuati siano errati o che altri vengano invece tralasciati. \\
    Individuazione & Uno degli obiettivi del team è di comunicare il più possibile con la proponente al fine di evitare problematiche riguardanti la corretta comprensione dei requisiti. Questo permette una più immediata rilevazione di errori nei requisiti o mancanza degli stessi. \\
    Occorrenza & Media \\
    Gravità & Media \\
    Piano di contingenza & Verranno discussi in dettaglio i requisiti che presentano più problemi di comprensione con la proponente.\\
\end{tabu}
\caption{Rischio: Comprensione dei requisiti}
\end{table}

\begin{table}[H]
\taburowcolors[2] 2{tableLineOne .. tableLineTwo}
\tabulinesep = 10pt
\everyrow{\tabucline[.4mm  white]{}}
\begin{tabu} to \textwidth { X[l,1.5] X[l,4] }
    \tableHeaderStyle
    Nome rischio & Cambiamenti nei requisiti \\
    Descrizione & È possibile che la proponente voglia apportare delle modifiche ai requisiti già individuati nel corso dello sviluppo. Questo potrebbe portare ad una modifica importante del documento Analisi dei Requisiti e creare problemi alla pianificazione delle attività. \\
    Individuazione & Si ritiene necessario lavorare da subito a stretto contatto con la proponente al fine di stabilire con certezza i requisiti, ed individuare il prima possibile eventuali modifiche.    \\
    Occorrenza & Bassa \\
    Gravità & Alta \\
    Piano di contingenza & In caso venga richiesto un cambiamento sostanzioso dei requisiti rispetto a quelli già individuati, questo verrà discusso da tutto il gruppo con la proponente al fine di minimizzare l'impatto negativo che ciò avrebbe sull'avanzamento. \\
\end{tabu}
\caption{Rischio: Cambiamenti nei requisiti}
\end{table}

\subsection{Rischi riguardanti le tecnologie}

\begin{table}[H]
\taburowcolors[2] 2{tableLineOne .. tableLineTwo}
\tabulinesep = 10pt
\everyrow{\tabucline[.4mm  white]{}}
\begin{tabu} to \textwidth { X[l,1.5] X[l,4] }
    \tableHeaderStyle
    Nome rischio & Tecnologie richieste per lo sviluppo \\
    Descrizione & Le tecnologie necessarie per lo sviluppo di alcune funzionalità sono relativamente recenti e in alcuni casi molto specifiche. Il tempo di apprendimento di queste tecnologie potrebbe causare ritardi nell'avanzamento. \\
    Individuazione & Il responsabile dovrà verificare la preparazione dei vari membri in relazione alle attività a loro assegnate. Se dovesse esserci una grave mancanza di competenza da parte di un membro questo dovrà farlo sapere al responsabile. \\
    Occorrenza & Alta \\
    Gravità & Media \\
    Piano di contingenza & In caso di gravi mancanze il responsabile potrebbe considerare di riservare più tempo allo studio di una specifica tecnologia, possibilmente includendo altri membri più esperti nell'apprendimento.\\
\end{tabu}
\caption{Rischio: Tecnologie richieste per lo sviluppo}
\end{table}

\subsection{Rischi riguardanti gli strumenti}

\begin{table}[H]
\taburowcolors[2] 2{tableLineOne .. tableLineTwo}
\tabulinesep = 10pt
\everyrow{\tabucline[.4mm  white]{}}
\begin{tabu} to \textwidth { X[l,1.5] X[l,4] }
    \tableHeaderStyle
    Nome rischio & Problematiche software \\
    Descrizione & Il gruppo fa riferimento a software di terze parti o servizi online, ad esempio \citgl{Github}, e un malfunzionamento di questi potrebbe portare a errori o alla perdita di materiale.  \\
    Individuazione & Difficile prevedere e rilevare questo genere di problematiche dato che spesso non dipendono in alcun modo dai membri del gruppo. \\
    Occorrenza & Bassa \\
    Gravità & Media \\
    Piano di contingenza & Gli strumenti scelti dal gruppo sono considerati molto affidabili, tuttavia data l'importanza dei dati prodotti dal team verranno effettuati backup frequenti per minimizzare la portata di eventuali errori.\\
\end{tabu}
\caption{Rischio: Problematiche software}
\end{table}

\begin{table}[H]
\taburowcolors[2] 2{tableLineOne .. tableLineTwo}
\tabulinesep = 10pt
\everyrow{\tabucline[.4mm  white]{}}
\begin{tabu} to \textwidth { X[l,1.5] X[l,4] }
    \tableHeaderStyle
    Nome rischio & Problematiche hardware \\
    Descrizione & Ogni membro utilizza il proprio pc personale per svolgere gli incarichi assegnati. Eventuali guasti potrebbero portare a ritardi nell'avanzamento o perdita di materiale. La portata di questa problematica è abbastanza ridotta data la disponibilità di computer secondari da parte dei membri del gruppo e il forte utilizzo di servizi di memorizzazione dati online. \\
    Individuazione & Il membro del gruppo che ha riscontrato un guasto nei propri strumenti dovrà avvisare il prima possibile il responsabile. \\
    Occorrenza & Bassa \\
    Gravità & Bassa \\
    Piano di contingenza & Ogni membro del gruppo dovrà effettuare frequenti backup di tutti i files relativi al progetto. In caso di temporanea impossibilità a procedere con i propri compiti dovuta a un guasto del pc il gruppo si impegnerà a fornire un dispositivo sostitutivo dove possibile.\\
\end{tabu}
\caption{Rischio: Problematiche hardware}
\end{table}
