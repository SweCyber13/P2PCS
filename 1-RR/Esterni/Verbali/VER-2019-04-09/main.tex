\documentclass[a4paper, oneside, openany, dvipsnames, table]{article}
\usepackage[utf8]{inputenc}


\usepackage{lmodern}
\usepackage{breakurl}
\usepackage[T1]{fontenc}
\usepackage[italian]{babel}
\usepackage{color, colortbl}
\usepackage{style}


%colori per tablle tabu
\definecolor{tableHeader}{RGB}{211, 47, 47}
\definecolor{tableLineOne}{RGB}{245, 245, 245}
\definecolor{tableLineTwo}{RGB}{224, 224, 224}
%define header tabelle tabu
\newcommand{\tableHeaderStyle}{
    \rowfont[c]{\leavevmode\color{white}\bfseries}
    \rowcolor{tableHeader}
}

\title{Verb-2019-04-09}
\author{Cyber13}
\date{March 2019}

\begin{document}
\begin{titlepage}
		\centering Università degli Studi di Padova
		\line(1,0){350}\\
		\vspace{1.2cm}
		\logo
		\vspace{1.0cm}
		\centering{\bfseries\LARGE Verbale esterno del 09/04/2019 \\}
		\vspace{0.5cm}
		\centering{\slshape\large Gruppo Cyber13 - Progetto P2PCS\\}
		\vspace{0.5cm}
		\centering{\bfseries Informazioni sul documento \\}
		\line(1,0){240}\\
		% compilare i campi per ogni documento
		\begin{tabular}{r|l}
			{\textbf{Versione}} 		& 1.0.0\\
			{\textbf{Data Redazione}} 	& 09/04/2019\\	% aggiornare la data
			{\textbf{Responsabile}} 	& Andrea Casagrande\\	% aggiornare la data
			{\textbf{Redazione}} 		& Fabio Garavello\\ 
			{\textbf{Verifica}} 			& Ilaria Rizzo\\ 
			{\textbf{Approvazione}} 		& Andrea Casagrande\\
			{\textbf{Uso}} 				& Interno\\
			{\textbf{Destinatari}} 	& GaiaGo\\	& Cyber13\\ & Prof. Tullio Vardanega\\ & Prof. Riccardo Cardin\\
			{\textbf{Mail di contatto}} 	& swe.cyber13@gmail.com\\
		\end{tabular}\\
	\end{titlepage}
	
	
\newpage
		\subfile{DiarioModifiche.tex}



\newpage
\tableofcontents
\newpage
\section{Informazioni sulla riunione}
\begin{itemize}
	\item \textbf{Luogo della riunione:} Aula 1BC45 Torre Tullio Levi Civita, Via Trieste, Padova;
	\item \textbf{Data della riunione:} 9 Aprile 2019;
	
	\item \textbf{Partecipanti della riunione:}
		\begin{itemize}
			\item Bira Daniel Mirel;
			\item Casagrande Andrea;
			\item Garavello Fabio;
			\item Pontecchiani Elena;
			\item Rizzo Ilaria;
			\item Squeri Matteo.
		\end{itemize}
\end{itemize}


	
	
	
\newpage
\section{Ordine del giorno}
Si è svolta una riunione con l'azienda proponente per chiarire dubbi sollevati dai componenti del gruppo. L'incaricato dell'azienda proponente Filippo Pretto ha raggiunto il team personalmente per poter mostrare a video e tramite slide, i punti ritenuti importanti per lo sviluppo del lavoro e per poter rispondere ad eventuali domande e chiarimenti.
In particolare sono stati trattati i seguenti punti:
\begin{itemize}
	\item Chiarimenti in merito all'utilizzo degli attori secondari. Nello specifico è stato mostrato a video il funzionamento servizio AWS (nel particolare il servizio Cognito);
	\item Chiarimenti in merito all'utilizzo della gamification. In particolare il gruppo ha esposto all'incaricato le idee e i requisiti individuati per lo sviluppo dell'applicazione al fine di avere un riscontro su quanto elaborato.
	 
		
\end{itemize}
	

\newpage
\section{Resoconto}
Al termine della riunione si è pervenuti alle seguenti conclusioni:	
\begin{itemize}
	\item  L'utilizzo di Aws (nel particolare Cognito) risulta essere consigliato dal proponente;
	\item È stato mostrato a video il funzionamento basilare di Cognito di Aws come strumento di accesso e registrazione per gli utenti della piattaforma;
	\item Sono stati analizzati i vari aspetti del modello Octalysis prendendo in considerazione i vari punti sui quali ci si appoggia per creare un tipo di applicazione più o meno orientata al gioco.
\end{itemize}
\newpage
\end{document}