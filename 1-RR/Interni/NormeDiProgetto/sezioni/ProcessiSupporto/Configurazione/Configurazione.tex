\subsubsection{Scopo}
Nella seguente sezione sono descritte le direttive riguardanti la configurazione degli strumenti di condivisione attraverso i quali sviluppare il materiale da consegnare.



\subsubsection{Aspettative}
Il gruppo si attende alle seguenti norme al fine di condividere tra i vari membri del team il materiale da consegnare in modo ordinato e tracciabile.




\subsubsection{Controllo di versione}
\paragraph{Descrizione}
~\\Si è scelto l'utilizzo della tecnologie Git per la raccolta e la condivisione dei file di documentazione e file non formali, opportunamente organizzati in \citgl{repository} separati.

\paragraph{Struttura delle repository}
~\\Durante la fase di RR son state create due cartelle, una contenente i file non formali chiamata "non formali" e una contenente la documentazione denominata "Documentazione" dove al suo interno è stata creata la cartella "RR" contenente i file di questa fase, organizzati secondo la seguente struttura:
\begin{itemize}
    \item \textbf{Documenti interni}
    \begin{itemize}
        \item Studio Di Fattibilità;
        \item Norme Di Progetto;
        \item Verbali interni.
    \end{itemize}
    \item \textbf{Documenti esterni}
    \begin{itemize}
        \item Piano di Progetto;
         \item Piano di Qualifica;
        \item Analisi dei Requisiti
        \item Glossario;
        \item Verbali esterni.
    \end{itemize}
\end{itemize}
All'interno delle cartelle son presenti i file finali in formato PDF  mentre i file \citgl{LaTex} divisi in sezioni con relativo diario modifiche restano salvati su un account condiviso da tutti i membri del gruppo sulla piattaforma \citgl{Overleaf}.
\paragraph{Branch}
~\\
Per il momento si è deciso di non fare uso dei \citgl{branch} in quanto per lo sviluppo della documentazione da presentare in fase RR si è stabilito di contenere tutto nel master branch. Successivamente verrà creato un branch per ogni membro del gruppo per la gestione del codice in parallelo in uno stesso documento.
\paragraph{Aggiornamento della repository}
~\\
\begin{itemize}
    \item Dare il comando ”git pull” per aggiornare il repository locale sulla base delle modifiche presente in remoto.
    \item Nel caso in cui si verifichino dei conflitti:
    \begin{itemize}
        \item Dare il comando ”git stash” per accantonare momentaneamente le modifiche apportate.
        \item Dare il comando ”git pull” per ottenere ed applicare i \citgl{commit} mancanti.
        \item Dare il comando ”git stash apply” per ripristinare le modifiche.
    \end{itemize}
    \item Dare il comando ”git add [files]” ,  che aggiungerà i file modificati e quelli nuovi specificati;
    \item Dare  il  comando  ”git  commit”  e  successivamente  riassumere  le  modifiche effettuate, in caso sia utile si può aggiungere un messaggio esteso di descrizione (aggiungendo al comando la dicitura -m "messaggio").
    \item Dare il comando ”git push” per completare l’operazione e fornire le modifiche agli altri membri del gruppo.
    
\end{itemize}

\subsubsection{Strumenti}

    \begin{itemize}
        \item\textbf{Client git:} 
        Secondo le preferenze dei membri del gruppo vengono usati principalmente due client \citgl{GitKraken} e \citgl{Git Bash}.
        \item\textbf{Server git:}
        Vista la conoscenza preliminare e l'affidabilità di \citgl{Github} si è scelto questo server.
        
    \end{itemize}

    
