\subsection{Informazioni sul capitolato}
    \begin{itemize}
        \item Nome del progetto: MegAlexa
        \item \textbf{Proponente}: ZERO12
        \item \textbf{Committente}: Prof Tullio Vardanega, Prof. Riccardo Cardin
    \end{itemize}
\subsection{Descrizione}
Il capitolato ha come scopo lo sviluppo di \citgl{skill per Alexa} di Amazon in grado di avviare dei \citgl{workflow} (routine di micro-funzioni) creati dagli utenti tramite interfaccia web o
mobile app per \citgl{iOS} e \citgl{Android}. I workflow sono completamente personalizzabili dagli utenti finali, in modo tale che una volta creati possano soddisfare appieno tutte
le loro esigenze senza che ci sia la necessità di dover fare manualmente la routine ogni volta. Per poter avviare i workflow creati
basta pronunciare "Alexa, esegui la routine" dove il nome del workflow viene assegnato nel momento della sua creazione. Il workflow può eseguire le micro-funzioni
anche in modo combinato, per esempio si può utilizzare una micro-funzione che legge un \citgl{feed rss} combinata con un'altra micro-funzione che filtra le notizie del feed rss interessate.
Per fare ciò è necessario creare una piattaforma (web e mobile) che offre delle micro-funzioni che possano essere collegate tra di loro creando il workflow voluto dall'utente.


\subsection{Studio del dominio}
     \begin{itemize}
        \item \textbf{Dominio applicativo}: Il capitolato fa riferimento all'interazione con i servizi di Amazon.
        \item \textbf{Dominio tecnologico}: 
            \begin{itemize}
                \item Amazon Web Services: Un insieme di servizi di \citgl{cloud} computing che compongono la piattaforma \citgl{on demand} offerta dall'azienda Amazon;
                \item API Gateway: Servizio completamente gestito che semplifica agli sviluppatori la creazione, la pubblicazione, la manutenzione, il monitoraggio e la protezione
		        delle \citgl{API} su qualsiasi scala;
                \item Lambda: Piattaforma che consente di eseguire codice senza dover effettuare il \citgl{provisioning} né gestire server;
                \item DynamoDB: Database che supporta i modelli di dati di tipo documento e di tipo chiave-valore che offre alte prestazioni a qualsiasi livello;
                \item Node.js: Piattaforma \citgl{open source} \citgl{event-driven} per l'esecuzione di codice JavaScript Server-side;
                \item Framework Node.js Express: Web \citgl{framework} per Node.js;
                \item Twitter bootstrap: Raccolta di strumenti liberi per la creazione di siti e applicazioni per il Web. Essa contiene modelli di progettazione basati su HTML, CSS
		        e JavaScript;
		        \item Alexa developer: Piattaforma di supporto per lo sviluppo di \citgl{skill Alexa};
        \end{itemize}
    \end{itemize}
\subsection{Esito finale}
    \begin{itemize}
        \item Aspetti positivi:
            \begin{itemize}
                \item Ritenuta interessante l'idea di creare una skill per Amazon Alexa;
                \item Lo studio e l'utilizzo dei servizi Amazon è reputato molto interessante dai componenti del gruppo;
                \item L'utilizzo dei framework Node.js Express e bootstrap facilitano lo sviluppo del codice;
                \item Interessante l'utilizzo del codice lato server.
            \end{itemize}
        \item Fattori di rischio:
            \begin{itemize}
                \item La maggior parte delle tecnologie risulta sconosciuta ai componenti del gruppo, inoltre si presentano innumerevoli \citgl{casi d'uso} da gestire per poter offrire all'utente
		        un'ampia gamma di micro-funzioni.
            \end{itemize}
        \item Conclusioni
            \begin{itemize}
                \item Sebbene il capitolato sia stato giudicato complessivamente molto interessante dai componenti del gruppo, il tempo necessario per la gestione di tutti i \citgl{casi d'uso}
		        e di tutte le micro-funzioni è stato considerato non sostenibile;
                \item Scelta: rigettato.
            \end{itemize}
    \end{itemize}